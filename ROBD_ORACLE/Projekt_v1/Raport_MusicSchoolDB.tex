\documentclass[12pt,a4paper]{article}
\usepackage[utf8]{inputenc}
\usepackage[T1]{fontenc}
\usepackage{lmodern}
\usepackage[polish]{babel}
\usepackage{geometry}
\usepackage{graphicx}
\usepackage{array}
\usepackage{booktabs}
\usepackage{enumitem}
\usepackage{amssymb}
\usepackage{tikz}
\usetikzlibrary{shapes.geometric, arrows.meta, positioning}
\usepackage[hidelinks]{hyperref}

\geometry{margin=2.5cm}

\begin{document}

% =============================================================================
% STRONA TYTUŁOWA
% =============================================================================
\begin{titlepage}
    \centering
    \vspace*{1cm}
    
    {\Large\textbf{Politechnika Łódzka}}\\[0.3cm]
    {\large Wydział Elektrotechniki, Elektroniki, Informatyki i Automatyki}\\[2cm]
    
    {\Huge\textbf{Szkoła Muzyczna}}\\[0.5cm]
    {\Large Obiektowa Baza Danych Oracle}\\[1cm]
    
    {\large Rozproszone i Obiektowe Bazy Danych}\\[3cm]
    
    \begin{tabular}{ll}
        \textbf{Autorzy:} & Igor Typiński (251237) \\
                          & Mateusz Mróz (251190) \\[0.5cm]
        \textbf{Grupa:}   & 5 \\[0.5cm]
        \textbf{Temat:}   & Szkoła muzyczna (z naciskiem na rozwój ucznia)
    \end{tabular}
    
    \vfill
    {\large Łódź, styczeń 2026}
\end{titlepage}

% =============================================================================
% SPIS TREŚCI
% =============================================================================
\tableofcontents
\newpage

% =============================================================================
% 1. OPIS PROJEKTU
% =============================================================================
\section{Opis projektu}

\subsection{Cel i zakres}
Projekt przedstawia obiektową bazę danych dla szkoły muzycznej, ze szczególnym uwzględnieniem śledzenia rozwoju uczniów. System umożliwia:
\begin{itemize}[nosep]
    \item Zarządzanie danymi uczniów i nauczycieli
    \item Planowanie i prowadzenie lekcji
    \item Ocenianie postępów uczniów w różnych obszarach muzycznych
    \item Przypisywanie nauczycieli do instrumentów
    \item Generowanie raportów i statystyk
\end{itemize}

\subsection{Technologia}
\begin{itemize}[nosep]
    \item Oracle Database
    \item Podejście obiektowo-relacyjne
    \item Język PL/SQL
\end{itemize}

% =============================================================================
% 2. REALIZACJA ZAŁOŻEŃ
% =============================================================================
\section{Realizacja założeń projektowych}

\subsection{Typy obiektowe}
W projekcie zdefiniowano 7 typów obiektowych:

\begin{table}[h]
\centering
\begin{tabular}{|l|l|p{6cm}|}
\hline
\textbf{Typ} & \textbf{Metody} & \textbf{Opis} \\
\hline
t\_instrument\_obj & 1 & Instrument muzyczny \\
t\_nauczyciel\_obj & 3 & Nauczyciel z listą instrumentów \\
t\_uczen\_obj & 3 & Uczeń z obliczaniem wieku \\
t\_kurs\_obj & 1 & Kurs nauki gry \\
t\_lekcja\_obj & 2 & Pojedyncza lekcja \\
t\_ocena\_obj & 2 & Ocena postępu ucznia \\
t\_lista\_instrumentow & -- & VARRAY(5) nazw instrumentów \\
\hline
\end{tabular}
\caption{Typy obiektowe w projekcie}
\end{table}

\subsection{Tabele obiektowe}
Utworzono 6 tabel obiektowych przechowujących dane:

\begin{table}[h]
\centering
\begin{tabular}{|l|l|l|}
\hline
\textbf{Tabela} & \textbf{Typ bazowy} & \textbf{Referencje (REF)} \\
\hline
t\_instrument & t\_instrument\_obj & -- \\
t\_nauczyciel & t\_nauczyciel\_obj & -- (zawiera VARRAY) \\
t\_uczen & t\_uczen\_obj & -- \\
t\_kurs & t\_kurs\_obj & ref\_instrument \\
t\_lekcja & t\_lekcja\_obj & ref\_uczen, ref\_nauczyciel, ref\_kurs \\
t\_ocena\_postepu & t\_ocena\_obj & ref\_uczen, ref\_nauczyciel \\
\hline
\end{tabular}
\caption{Tabele obiektowe i ich referencje}
\end{table}

\subsection{Referencje (REF/DEREF)}
W projekcie zastosowano 6 referencji do modelowania relacji między obiektami:
\begin{itemize}[nosep]
    \item Kurs wskazuje na instrument, którego dotyczy
    \item Lekcja wskazuje na ucznia, nauczyciela i kurs
    \item Ocena wskazuje na ucznia i nauczyciela wystawiającego
\end{itemize}

Dzięki DEREF możliwe jest odwołanie się do atrybutów i metod obiektu wskazywanego, np. pobranie pełnych danych ucznia z poziomu lekcji.

\subsection{Kolekcja VARRAY}
Typ \texttt{t\_lista\_instrumentow} jako VARRAY(5) przechowuje listę instrumentów, których może uczyć nauczyciel. Ograniczenie do 5 elementów wynika z założenia, że nauczyciel specjalizuje się w kilku instrumentach.

\subsection{Pakiety PL/SQL}
Logika biznesowa zaimplementowana w 3 pakietach:

\begin{table}[h]
\centering
\begin{tabular}{|l|l|p{7cm}|}
\hline
\textbf{Pakiet} & \textbf{Procedur} & \textbf{Główne funkcjonalności} \\
\hline
pkg\_uczen & 5 & Dodawanie uczniów, lista, średnia ocen, filtrowanie wg wieku \\
pkg\_lekcja & 6 & Planowanie, oznaczanie, odwoływanie lekcji, raporty \\
pkg\_ocena & 5 & Dodawanie ocen, raport postępu, porównanie uczniów \\
\hline
\end{tabular}
\caption{Pakiety PL/SQL}
\end{table}

\subsection{Kursory}
W projekcie wykorzystano trzy rodzaje kursorów:
\begin{itemize}[nosep]
    \item \textbf{Jawne} -- deklarowane z CURSOR, obsługiwane przez OPEN/FETCH/CLOSE
    \item \textbf{Niejawne (FOR)} -- uproszczona składnia FOR rec IN (SELECT...)
    \item \textbf{REF CURSOR} -- dynamiczne kursory zwracane przez funkcje
\end{itemize}

\subsection{Wyzwalacze (Triggery)}
Zdefiniowano 5 wyzwalaczy:

\begin{table}[h]
\centering
\begin{tabular}{|l|l|p{6cm}|}
\hline
\textbf{Trigger} & \textbf{Typ} & \textbf{Działanie} \\
\hline
trg\_lekcja\_walidacja & BEFORE & Walidacja daty, godziny, konfliktów \\
trg\_ocena\_audit & AFTER & Logowanie operacji na ocenach \\
trg\_uczen\_przed\_usunieciem & BEFORE & Ochrona ucznia z lekcjami \\
trg\_nauczyciel\_data & BEFORE & Domyślna data zatrudnienia \\
trg\_kurs\_cena\_audit & AFTER & Logowanie zmian cen \\
\hline
\end{tabular}
\caption{Wyzwalacze w projekcie}
\end{table}

\subsection{Obsługa błędów}
Zastosowano mechanizmy obsługi wyjątków:
\begin{itemize}[nosep]
    \item Bloki EXCEPTION z obsługą standardowych wyjątków
    \item RAISE\_APPLICATION\_ERROR dla błędów biznesowych
    \item Własne kody błędów (-20001 do -20021)
\end{itemize}

% =============================================================================
% 3. PRZYJĘTE OGRANICZENIA
% =============================================================================
\section{Przyjęte ograniczenia}

\begin{enumerate}[nosep]
    \item Minimalny wiek ucznia: 5 lat
    \item Nauczyciel może uczyć maksymalnie 5 instrumentów (VARRAY)
    \item Oceny w skali 1--6 (polska skala szkolna)
    \item Lekcje tylko w godzinach 08:00--20:00
    \item Czas trwania lekcji: 30, 45, 60 lub 90 minut
    \item Kategorie instrumentów: dęte, strunowe, perkusyjne, klawiszowe
    \item Obszary oceny: technika, teoria, słuch, rytm, interpretacja
    \item Poziomy kursów: początkujący, średni, zaawansowany
    \item Statusy lekcji: zaplanowana, odbyta, odwołana
    \item Uczeń nie może mieć dwóch lekcji o tej samej godzinie
\end{enumerate}

% =============================================================================
% 4. ROLE UŻYTKOWNIKÓW
% =============================================================================
\section{Role użytkowników}

\subsection{Administrator (rola\_admin)}
Pełny dostęp do systemu:
\begin{itemize}[nosep]
    \item Zarządzanie wszystkimi danymi (CRUD na wszystkich tabelach)
    \item Dostęp do logów audytowych
    \item Wykonywanie wszystkich pakietów
    \item Zarządzanie użytkownikami i uprawnieniami
\end{itemize}

\subsection{Nauczyciel (rola\_nauczyciel)}
Prowadzenie lekcji i ocenianie:
\begin{itemize}[nosep]
    \item Odczyt danych uczniów, kursów, instrumentów
    \item Aktualizacja statusu/tematu/uwag lekcji
    \item Dodawanie ocen postępu
    \item Generowanie raportów dziennych
\end{itemize}

\subsection{Sekretariat (rola\_sekretariat)}
Zarządzanie harmonogramem:
\begin{itemize}[nosep]
    \item Pełne zarządzanie danymi uczniów
    \item Planowanie, modyfikowanie i odwoływanie lekcji
    \item Odczyt ocen i danych nauczycieli
    \item Generowanie list i raportów
\end{itemize}

\begin{table}[h]
\centering
\begin{tabular}{|l|c|c|c|}
\hline
\textbf{Funkcjonalność} & \textbf{Admin} & \textbf{Nauczyciel} & \textbf{Sekretariat} \\
\hline
Zarządzanie uczniami & \checkmark & -- & \checkmark \\
Zarządzanie nauczycielami & \checkmark & -- & -- \\
Planowanie lekcji & \checkmark & -- & \checkmark \\
Aktualizacja lekcji & \checkmark & \checkmark & \checkmark \\
Dodawanie ocen & \checkmark & \checkmark & -- \\
Odczyt ocen & \checkmark & \checkmark & \checkmark \\
Logi audytowe & \checkmark & -- & -- \\
\hline
\end{tabular}
\caption{Macierz uprawnień}
\end{table}

% =============================================================================
% 5. DIAGRAM RELACJI OBIEKTÓW
% =============================================================================
\section{Diagram relacji obiektów}

\begin{center}
\begin{tikzpicture}[
    node distance=2cm,
    every node/.style={font=\small},
    type/.style={rectangle, draw, fill=blue!20, minimum width=3cm, minimum height=1cm, align=center},
    varray/.style={rectangle, draw, fill=green!20, minimum width=2.5cm, minimum height=0.8cm, align=center},
    arrow/.style={-{Stealth}, thick}
]

% Typy główne
\node[type] (instrument) {T\_INSTRUMENT\\{\scriptsize(klawiszowe, strunowe...)}};
\node[type, right=3cm of instrument] (kurs) {T\_KURS\\{\scriptsize(poziom, cena)}};
\node[type, below=2cm of instrument] (nauczyciel) {T\_NAUCZYCIEL\\{\scriptsize(imię, nazwisko, staż)}};
\node[type, right=3cm of nauczyciel] (lekcja) {T\_LEKCJA\\{\scriptsize(data, godzina, status)}};
\node[type, below=2cm of nauczyciel] (uczen) {T\_UCZEŃ\\{\scriptsize(imię, nazwisko, wiek)}};
\node[type, right=3cm of uczen] (ocena) {T\_OCENA\_POSTĘPU\\{\scriptsize(ocena 1-6, obszar)}};

% VARRAY
\node[varray, left=1.5cm of nauczyciel] (varray) {VARRAY\\{\scriptsize lista\_instrumentów}};

% Strzałki REF
\draw[arrow] (kurs) -- node[above, font=\scriptsize] {REF} (instrument);
\draw[arrow] (lekcja) -- node[above, font=\scriptsize] {REF} (kurs);
\draw[arrow] (lekcja) -- node[left, font=\scriptsize] {REF} (nauczyciel);
\draw[arrow] (lekcja) -- node[below, font=\scriptsize] {REF} (uczen);
\draw[arrow] (ocena) -- node[left, font=\scriptsize] {REF} (nauczyciel);
\draw[arrow] (ocena) -- node[above, font=\scriptsize] {REF} (uczen);

% VARRAY
\draw[arrow, dashed] (nauczyciel) -- (varray);

\end{tikzpicture}
\end{center}

\vspace{0.5cm}

\textbf{Legenda:}
\begin{itemize}[nosep]
    \item Strzałka ciągła -- referencja REF (wskaźnik do obiektu)
    \item Strzałka przerywana -- zawiera VARRAY (kolekcja)
    \item Niebieskie prostokąty -- typy obiektowe / tabele
    \item Zielony prostokąt -- typ kolekcji VARRAY
\end{itemize}

% =============================================================================
% 6. PODSUMOWANIE
% =============================================================================
\section{Podsumowanie}

Projekt obiektowej bazy danych dla szkoły muzycznej spełnia wszystkie wymagania:

\begin{table}[h]
\centering
\begin{tabular}{|p{8cm}|c|}
\hline
\textbf{Wymaganie} & \textbf{Realizacja} \\
\hline
Definicje typów obiektowych z metodami & 7 typów, 12 metod \\
Tabele obiektowe (wierszowe) & 6 tabel \\
Referencja i dereferencja & 6 REF, DEREF \\
Wstawianie danych z referencją & Tak \\
VARRAY do relacji 1:N & t\_lista\_instrumentow \\
Pakiety PL/SQL & 3 pakiety \\
Kursory i REF kursory & 3 typy kursorów \\
Obsługa błędów & EXCEPTION, RAISE \\
Wyzwalacze & 5 triggerów \\
Role użytkowników & 3 role \\
\hline
\end{tabular}
\caption{Realizacja wymagań projektowych}
\end{table}

\end{document}
