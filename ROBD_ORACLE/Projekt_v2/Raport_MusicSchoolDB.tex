\documentclass[12pt,a4paper]{article}
\usepackage[utf8]{inputenc}
\usepackage[T1]{fontenc}
\usepackage{lmodern}
\usepackage[polish]{babel}
\usepackage{geometry}
\usepackage{graphicx}
\usepackage{array}
\usepackage{booktabs}
\usepackage{enumitem}
\usepackage{amssymb}
\usepackage{tikz}
\usetikzlibrary{shapes.geometric, arrows.meta, positioning}
\usepackage[hidelinks]{hyperref}
\usepackage{longtable}

\geometry{margin=2.5cm}

\begin{document}

% =============================================================================
% STRONA TYTUŁOWA
% =============================================================================
\begin{titlepage}
    \centering
    \vspace*{1cm}
    
    {\Large\textbf{Politechnika Łódzka}}\\[0.3cm]
    {\large Wydział Elektrotechniki, Elektroniki, Informatyki i Automatyki}\\[2cm]
    
    {\Huge\textbf{Szkoła Muzyczna}}\\[0.5cm]
    {\Large Obiektowa Baza Danych Oracle}\\[1cm]
    
    {\large Rozproszone i Obiektowe Bazy Danych}\\[3cm]
    
    \begin{tabular}{ll}
        \textbf{Autorzy:} & Igor Typiński (251237) \\
                          & Mateusz Mróz (251190) \\[0.5cm]
        \textbf{Grupa:}   & 5 \\[0.5cm]
        \textbf{Temat:}   & Szkoła muzyczna (z naciskiem na rozwój ucznia)
    \end{tabular}
    
    \vfill
    {\large Łódź, styczeń 2026}
\end{titlepage}

% =============================================================================
% SPIS TREŚCI
% =============================================================================
\tableofcontents
\newpage

% =============================================================================
% 1. OPIS PROJEKTU
% =============================================================================
\section{Opis projektu}

\subsection{Cel i zakres}
Projekt przedstawia obiektową bazę danych dla szkoły muzycznej, ze szczególnym uwzględnieniem śledzenia rozwoju uczniów. Szkoła zajmuje się wyłącznie nauką muzyki. System umożliwia:
\begin{itemize}[nosep]
    \item Zarządzanie danymi uczniów i nauczycieli
    \item Planowanie i prowadzenie lekcji w ramach semestrów
    \item Rezerwację sal lekcyjnych z odpowiednim wyposażeniem
    \item Ocenianie postępów uczniów w różnych obszarach muzycznych
    \item Przypisywanie nauczycieli do instrumentów
    \item Kontrolę limitów obciążenia nauczycieli i uczniów
    \item Uwzględnienie grafiku dzieci uczęszczających do zwykłej szkoły
    \item Generowanie raportów i statystyk
\end{itemize}

\subsection{Technologia}
\begin{itemize}[nosep]
    \item Oracle Database
    \item Podejście obiektowo-relacyjne
    \item Język PL/SQL
    \item Typy obiektowe z metodami
    \item Referencje REF/DEREF
    \item Kolekcje VARRAY
\end{itemize}

% =============================================================================
% 2. TYPY OBIEKTOWE
% =============================================================================
\section{Typy obiektowe}

W projekcie zdefiniowano 9 typów obiektowych z łącznie 15 metodami.

\begin{table}[h]
\centering
\begin{tabular}{|l|c|p{6cm}|}
\hline
\textbf{Typ} & \textbf{Metody} & \textbf{Opis} \\
\hline
t\_uczen\_obj & 3 & Uczeń szkoły muzycznej \\
t\_nauczyciel\_obj & 3 & Nauczyciel z listą instrumentów \\
t\_kurs\_obj & 2 & Kurs nauki gry na instrumencie \\
t\_zapis\_obj & 2 & Zapis ucznia na kurs \\
t\_lekcja\_obj & 2 & Pojedyncza lekcja muzyki \\
t\_ocena\_obj & 1 & Ocena postępu ucznia \\
t\_sala\_obj & 1 & Sala lekcyjna z wyposażeniem \\
t\_semestr\_obj & 2 & Semestr akademicki \\
t\_lista\_instrumentow & -- & VARRAY(5) nazw instrumentów \\
\hline
\end{tabular}
\caption{Typy obiektowe w projekcie}
\end{table}

\subsection{Szczegółowy opis typów}

\subsubsection{t\_uczen\_obj}
Reprezentuje ucznia szkoły muzycznej.

\textbf{Atrybuty:}
\begin{itemize}[nosep]
    \item id\_ucznia (NUMBER) -- unikalny identyfikator
    \item imie (VARCHAR2(50)) -- imię ucznia
    \item nazwisko (VARCHAR2(50)) -- nazwisko ucznia
    \item data\_urodzenia (DATE) -- data urodzenia
    \item email (VARCHAR2(100)) -- adres email (unikalny)
    \item telefon (VARCHAR2(15)) -- numer telefonu
    \item data\_rejestracji (DATE) -- data zapisu do szkoły
\end{itemize}

\textbf{Metody:}
\begin{itemize}[nosep]
    \item wiek() RETURN NUMBER -- oblicza aktualny wiek ucznia
    \item czy\_pelnoletni() RETURN VARCHAR2 -- zwraca 'TAK' lub 'NIE'
    \item pelne\_dane() RETURN VARCHAR2 -- zwraca imię i nazwisko
\end{itemize}

\subsubsection{t\_nauczyciel\_obj}
Reprezentuje nauczyciela szkoły muzycznej.

\textbf{Atrybuty:}
\begin{itemize}[nosep]
    \item id\_nauczyciela (NUMBER) -- unikalny identyfikator
    \item imie (VARCHAR2(50)) -- imię nauczyciela
    \item nazwisko (VARCHAR2(50)) -- nazwisko nauczyciela
    \item data\_urodzenia (DATE) -- data urodzenia
    \item email (VARCHAR2(100)) -- adres email (unikalny)
    \item telefon (VARCHAR2(15)) -- numer telefonu
    \item instrumenty (t\_lista\_instrumentow) -- VARRAY instrumentów
    \item wyksztalcenie (VARCHAR2(50)) -- poziom wykształcenia
    \item lata\_doswiadczenia (NUMBER) -- staż pracy
\end{itemize}

\textbf{Metody:}
\begin{itemize}[nosep]
    \item pelne\_dane() RETURN VARCHAR2 -- zwraca imię i nazwisko
    \item liczba\_instrumentow() RETURN NUMBER -- liczba instrumentów w VARRAY
    \item czy\_senior() RETURN VARCHAR2 -- 'TAK' jeśli staż >= 10 lat
\end{itemize}

\subsubsection{t\_kurs\_obj}
Reprezentuje kurs nauki gry na instrumencie.

\textbf{Atrybuty:}
\begin{itemize}[nosep]
    \item id\_kursu (NUMBER) -- unikalny identyfikator
    \item nazwa (VARCHAR2(100)) -- nazwa kursu
    \item typ (VARCHAR2(20)) -- 'indywidualny' lub 'grupowy'
    \item opis (VARCHAR2(500)) -- opis kursu
    \item cena\_za\_lekcje (NUMBER(10,2)) -- cena jednej lekcji
    \item domyslny\_czas (NUMBER) -- domyślny czas trwania w minutach
\end{itemize}

\textbf{Metody:}
\begin{itemize}[nosep]
    \item czy\_indywidualny() RETURN VARCHAR2 -- 'TAK' lub 'NIE'
    \item info() RETURN VARCHAR2 -- zwraca nazwę z typem i ceną
\end{itemize}

\subsubsection{t\_zapis\_obj}
Reprezentuje zapis ucznia na kurs.

\textbf{Atrybuty:}
\begin{itemize}[nosep]
    \item id\_zapisu (NUMBER) -- unikalny identyfikator
    \item data\_zapisu (DATE) -- data dokonania zapisu
    \item status (VARCHAR2(20)) -- 'aktywny', 'zawieszony', 'zakończony'
    \item ref\_uczen (REF t\_uczen\_obj) -- referencja do ucznia
    \item ref\_kurs (REF t\_kurs\_obj) -- referencja do kursu
    \item ref\_nauczyciel (REF t\_nauczyciel\_obj) -- referencja do nauczyciela
\end{itemize}

\textbf{Metody:}
\begin{itemize}[nosep]
    \item czy\_aktywny() RETURN VARCHAR2 -- 'TAK' lub 'NIE'
    \item dni\_od\_zapisu() RETURN NUMBER -- liczba dni od zapisu
\end{itemize}

\subsubsection{t\_lekcja\_obj}
Reprezentuje pojedynczą lekcję muzyki.

\textbf{Atrybuty:}
\begin{itemize}[nosep]
    \item id\_lekcji (NUMBER) -- unikalny identyfikator
    \item data\_lekcji (DATE) -- data lekcji
    \item godzina\_start (VARCHAR2(5)) -- godzina rozpoczęcia (HH:MM)
    \item czas\_trwania (NUMBER) -- czas w minutach (30, 45, 60, 90)
    \item temat (VARCHAR2(200)) -- temat lekcji
    \item uwagi (VARCHAR2(500)) -- uwagi nauczyciela
    \item status (VARCHAR2(20)) -- 'zaplanowana', 'odbyta', 'odwołana'
    \item ref\_uczen (REF t\_uczen\_obj) -- referencja do ucznia
    \item ref\_nauczyciel (REF t\_nauczyciel\_obj) -- referencja do nauczyciela
    \item ref\_kurs (REF t\_kurs\_obj) -- referencja do kursu
    \item ref\_sala (REF t\_sala\_obj) -- referencja do sali
\end{itemize}

\textbf{Metody:}
\begin{itemize}[nosep]
    \item czy\_odbyta() RETURN VARCHAR2 -- 'TAK' lub 'NIE'
    \item krotki\_opis() RETURN VARCHAR2 -- data, godzina, status
\end{itemize}

\subsubsection{t\_ocena\_obj}
Reprezentuje ocenę postępu ucznia.

\textbf{Atrybuty:}
\begin{itemize}[nosep]
    \item id\_oceny (NUMBER) -- unikalny identyfikator
    \item data\_oceny (DATE) -- data wystawienia oceny
    \item ocena (NUMBER(1)) -- wartość 1--6
    \item komentarz (VARCHAR2(500)) -- komentarz nauczyciela
    \item obszar (VARCHAR2(50)) -- obszar oceny (technika, słuch, rytm...)
    \item ref\_uczen (REF t\_uczen\_obj) -- referencja do ucznia
    \item ref\_nauczyciel (REF t\_nauczyciel\_obj) -- referencja do nauczyciela
\end{itemize}

\textbf{Metody:}
\begin{itemize}[nosep]
    \item poziom\_slowny() RETURN VARCHAR2 -- 'niedostateczny' do 'celujący'
\end{itemize}

\subsubsection{t\_sala\_obj}
Reprezentuje salę lekcyjną z wyposażeniem.

\textbf{Atrybuty:}
\begin{itemize}[nosep]
    \item id\_sali (NUMBER) -- unikalny identyfikator
    \item nazwa (VARCHAR2(50)) -- nazwa sali (unikalna)
    \item pojemnosc (NUMBER) -- maksymalna liczba osób
    \item ma\_fortepian (CHAR(1)) -- 'T' lub 'N'
    \item ma\_perkusje (CHAR(1)) -- 'T' lub 'N'
    \item opis (VARCHAR2(200)) -- dodatkowy opis wyposażenia
\end{itemize}

\textbf{Metody:}
\begin{itemize}[nosep]
    \item opis\_pelny() RETURN VARCHAR2 -- nazwa z informacją o wyposażeniu
\end{itemize}

\subsubsection{t\_semestr\_obj}
Reprezentuje semestr akademicki.

\textbf{Atrybuty:}
\begin{itemize}[nosep]
    \item id\_semestru (NUMBER) -- unikalny identyfikator
    \item nazwa (VARCHAR2(50)) -- nazwa semestru (unikalna)
    \item data\_od (DATE) -- data rozpoczęcia
    \item data\_do (DATE) -- data zakończenia
    \item czy\_aktywny (CHAR(1)) -- 'T' lub 'N'
\end{itemize}

\textbf{Metody:}
\begin{itemize}[nosep]
    \item czy\_w\_trakcie() RETURN VARCHAR2 -- 'TAK' jeśli SYSDATE w zakresie
    \item dni\_do\_konca() RETURN NUMBER -- liczba dni do końca semestru
\end{itemize}

\subsubsection{t\_lista\_instrumentow}
Kolekcja VARRAY przechowująca nazwy instrumentów.

\begin{verbatim}
CREATE TYPE t_lista_instrumentow AS VARRAY(5) OF VARCHAR2(50);
\end{verbatim}

Ograniczenie do 5 elementów wynika z założenia, że nauczyciel specjalizuje się w kilku instrumentach.

% =============================================================================
% 3. TABELE OBIEKTOWE
% =============================================================================
\section{Tabele obiektowe}

Utworzono 8 tabel obiektowych przechowujących dane.

\begin{table}[h]
\centering
\begin{tabular}{|l|l|l|}
\hline
\textbf{Tabela} & \textbf{Typ bazowy} & \textbf{Referencje (REF)} \\
\hline
t\_uczen & t\_uczen\_obj & -- \\
t\_nauczyciel & t\_nauczyciel\_obj & -- (zawiera VARRAY) \\
t\_kurs & t\_kurs\_obj & -- \\
t\_zapis & t\_zapis\_obj & ref\_uczen, ref\_kurs, ref\_nauczyciel \\
t\_lekcja & t\_lekcja\_obj & ref\_uczen, ref\_nauczyciel, ref\_kurs, ref\_sala \\
t\_ocena\_postepu & t\_ocena\_obj & ref\_uczen, ref\_nauczyciel \\
t\_sala & t\_sala\_obj & -- \\
t\_semestr & t\_semestr\_obj & -- \\
\hline
\end{tabular}
\caption{Tabele obiektowe i ich referencje}
\end{table}

\subsection{Szczegółowy opis tabel}

\subsubsection{t\_uczen}
Przechowuje dane uczniów szkoły muzycznej.
\begin{itemize}[nosep]
    \item Klucz główny: id\_ucznia
    \item Ograniczenie UNIQUE: email
    \item Indeks: idx\_uczen\_nazwisko (nazwisko)
\end{itemize}

\subsubsection{t\_nauczyciel}
Przechowuje dane nauczycieli wraz z listą instrumentów (VARRAY).
\begin{itemize}[nosep]
    \item Klucz główny: id\_nauczyciela
    \item Ograniczenie UNIQUE: email
    \item Zawiera kolekcję t\_lista\_instrumentow
\end{itemize}

\subsubsection{t\_kurs}
Przechowuje ofertę kursów szkoły.
\begin{itemize}[nosep]
    \item Klucz główny: id\_kursu
    \item CHECK: typ IN ('indywidualny', 'grupowy')
    \item CHECK: cena\_za\_lekcje > 0
    \item CHECK: domyslny\_czas IN (30, 45, 60, 90)
\end{itemize}

\subsubsection{t\_zapis}
Przechowuje zapisy uczniów na kursy.
\begin{itemize}[nosep]
    \item Klucz główny: id\_zapisu
    \item CHECK: status IN ('aktywny', 'zawieszony', 'zakończony')
    \item Referencje do: t\_uczen, t\_kurs, t\_nauczyciel
\end{itemize}

\subsubsection{t\_lekcja}
Przechowuje zaplanowane i odbyte lekcje.
\begin{itemize}[nosep]
    \item Klucz główny: id\_lekcji
    \item CHECK: status IN ('zaplanowana', 'odbyta', 'odwołana')
    \item CHECK: czas\_trwania IN (30, 45, 60, 90)
    \item Referencje do: t\_uczen, t\_nauczyciel, t\_kurs, t\_sala
    \item Indeks: idx\_lekcja\_data (data\_lekcji)
\end{itemize}

\subsubsection{t\_ocena\_postepu}
Przechowuje oceny postępów uczniów.
\begin{itemize}[nosep]
    \item Klucz główny: id\_oceny
    \item CHECK: ocena BETWEEN 1 AND 6
    \item CHECK: obszar IN ('technika', 'teoria', 'sluch', 'rytm', 'interpretacja', 'ogolna')
    \item Referencje do: t\_uczen, t\_nauczyciel
\end{itemize}

\subsubsection{t\_sala}
Przechowuje informacje o salach lekcyjnych.
\begin{itemize}[nosep]
    \item Klucz główny: id\_sali
    \item Ograniczenie UNIQUE: nazwa
    \item CHECK: pojemnosc > 0
    \item CHECK: ma\_fortepian IN ('T', 'N')
    \item CHECK: ma\_perkusje IN ('T', 'N')
\end{itemize}

\subsubsection{t\_semestr}
Przechowuje informacje o semestrach.
\begin{itemize}[nosep]
    \item Klucz główny: id\_semestru
    \item Ograniczenie UNIQUE: nazwa
    \item CHECK: czy\_aktywny IN ('T', 'N')
    \item CHECK: data\_do > data\_od
\end{itemize}

\subsection{Referencje (REF/DEREF)}
W projekcie zastosowano 8 referencji do modelowania relacji między obiektami:
\begin{itemize}[nosep]
    \item Zapis wskazuje na ucznia, kurs i nauczyciela prowadzącego
    \item Lekcja wskazuje na ucznia, nauczyciela, kurs i salę
    \item Ocena wskazuje na ucznia i nauczyciela wystawiającego
\end{itemize}

Dzięki DEREF możliwe jest odwołanie się do atrybutów i metod obiektu wskazywanego:
\begin{verbatim}
SELECT DEREF(l.ref_uczen).pelne_dane() AS uczen,
       DEREF(l.ref_sala).nazwa AS sala
FROM t_lekcja l;
\end{verbatim}

\subsection{Sekwencje}
Utworzono 8 sekwencji do generowania identyfikatorów:
\begin{itemize}[nosep]
    \item seq\_uczen, seq\_nauczyciel, seq\_kurs, seq\_zapis
    \item seq\_lekcja, seq\_ocena, seq\_sala, seq\_semestr
\end{itemize}

% =============================================================================
% 4. PAKIETY PL/SQL
% =============================================================================
\section{Pakiety PL/SQL}

Logika biznesowa zaimplementowana w 5 pakietach z łącznie 26 procedurami i funkcjami.

\begin{table}[h]
\centering
\begin{tabular}{|l|c|p{6cm}|}
\hline
\textbf{Pakiet} & \textbf{Procedur} & \textbf{Główne funkcjonalności} \\
\hline
pkg\_semestr & 4 & Zarządzanie semestrami \\
pkg\_sala & 4 & Zarządzanie salami lekcyjnymi \\
pkg\_uczen & 5 & Zarządzanie uczniami \\
pkg\_lekcja & 8 & Planowanie i zarządzanie lekcjami \\
pkg\_ocena & 5 & Ocenianie postępów uczniów \\
\hline
\end{tabular}
\caption{Pakiety PL/SQL}
\end{table}

\subsection{pkg\_semestr}
Pakiet do zarządzania semestrami akademickimi.

\textbf{Procedury i funkcje:}
\begin{itemize}[nosep]
    \item utworz\_semestr(p\_nazwa, p\_data\_od, p\_data\_do, p\_aktywny) -- tworzy nowy semestr
    \item aktywuj\_semestr(p\_id\_semestru) -- aktywuje wybrany semestr (dezaktywuje pozostałe)
    \item pobierz\_aktywny\_semestr() RETURN t\_semestr\_obj -- zwraca aktywny semestr
    \item info\_semestr() -- wyświetla informacje o aktywnym semestrze
\end{itemize}

\subsection{pkg\_sala}
Pakiet do zarządzania salami lekcyjnymi.

\textbf{Procedury i funkcje:}
\begin{itemize}[nosep]
    \item dodaj\_sale(p\_nazwa, p\_pojemnosc, p\_ma\_fortepian, p\_ma\_perkusje, p\_opis) -- dodaje salę
    \item sprawdz\_dostepnosc(p\_id\_sali, p\_data, p\_godzina) RETURN VARCHAR2 -- 'WOLNA' lub 'ZAJETA'
    \item lista\_sal() -- wyświetla listę wszystkich sal
    \item sale\_wolne(p\_data, p\_godzina) RETURN SYS\_REFCURSOR -- zwraca wolne sale
\end{itemize}

\subsection{pkg\_uczen}
Pakiet do zarządzania uczniami.

\textbf{Stałe:}
\begin{itemize}[nosep]
    \item c\_min\_wiek CONSTANT NUMBER := 5 -- minimalny wiek ucznia
\end{itemize}

\textbf{Procedury i funkcje:}
\begin{itemize}[nosep]
    \item dodaj\_ucznia(p\_imie, p\_nazwisko, p\_data\_urodzenia, p\_email, p\_telefon) -- dodaje ucznia
    \item liczba\_uczniow() RETURN NUMBER -- zwraca liczbę uczniów
    \item uczniowie\_wiek(p\_wiek\_min, p\_wiek\_max) RETURN SYS\_REFCURSOR -- filtruje wg wieku
    \item lista\_uczniow() -- wyświetla listę uczniów (kursor jawny)
    \item srednia\_ocen(p\_id\_ucznia) RETURN NUMBER -- oblicza średnią ocen
\end{itemize}

\subsection{pkg\_lekcja}
Pakiet do planowania i zarządzania lekcjami.

\textbf{Procedury i funkcje:}
\begin{itemize}[nosep]
    \item zaplanuj\_lekcje(p\_id\_ucznia, p\_id\_nauczyciela, p\_id\_kursu, p\_id\_sali, p\_data, p\_godzina, p\_czas\_trwania) -- planuje lekcję
    \item oznacz\_odbyta(p\_id\_lekcji, p\_temat, p\_uwagi) -- oznacza lekcję jako odbytą
    \item odwolaj\_lekcje(p\_id\_lekcji) -- odwołuje lekcję
    \item lekcje\_ucznia(p\_id\_ucznia, p\_miesiac) RETURN SYS\_REFCURSOR -- lekcje ucznia w miesiącu
    \item lekcje\_tygodniowo(p\_id\_nauczyciela) RETURN NUMBER -- liczba lekcji w tygodniu
    \item raport\_dzienny(p\_data) -- wyświetla lekcje na dany dzień
    \item sprawdz\_dostepnosc\_kompleksowa(p\_id\_ucznia, p\_id\_nauczyciela, p\_id\_sali, p\_data, p\_godzina) RETURN VARCHAR2 -- sprawdza konflikty
    \item statystyki\_nauczyciela(p\_id\_nauczyciela, p\_data) -- wyświetla obciążenie nauczyciela
\end{itemize}

\subsection{pkg\_ocena}
Pakiet do oceniania postępów uczniów.

\textbf{Procedury i funkcje:}
\begin{itemize}[nosep]
    \item dodaj\_ocene(p\_id\_ucznia, p\_id\_nauczyciela, p\_ocena, p\_obszar, p\_komentarz) -- dodaje ocenę
    \item ostatnie\_oceny(p\_id\_ucznia, p\_limit) RETURN SYS\_REFCURSOR -- ostatnie N ocen
    \item raport\_postepu(p\_id\_ucznia) -- wyświetla raport postępu ucznia
    \item srednia\_obszar(p\_id\_ucznia, p\_obszar) RETURN NUMBER -- średnia w danym obszarze
    \item porownaj\_uczniow(p\_id\_ucznia\_1, p\_id\_ucznia\_2) -- porównuje dwóch uczniów
\end{itemize}

\subsection{Kursory}
W projekcie wykorzystano trzy rodzaje kursorów:
\begin{itemize}[nosep]
    \item \textbf{Jawne} -- deklarowane z CURSOR, obsługiwane przez OPEN/FETCH/CLOSE (pkg\_uczen.lista\_uczniow)
    \item \textbf{Niejawne (FOR)} -- uproszczona składnia FOR rec IN (SELECT...) (pkg\_lekcja.raport\_dzienny)
    \item \textbf{REF CURSOR} -- dynamiczne kursory zwracane przez funkcje (SYS\_REFCURSOR)
\end{itemize}

% =============================================================================
% 5. WYZWALACZE (TRIGGERY)
% =============================================================================
\section{Wyzwalacze (Triggery)}

Zdefiniowano 16 wyzwalaczy podzielonych na kategorie.

\subsection{Triggery walidacyjne}

\begin{table}[h]
\centering
\begin{tabular}{|l|l|p{5cm}|}
\hline
\textbf{Trigger} & \textbf{Typ} & \textbf{Działanie} \\
\hline
trg\_uczen\_wiek & BEFORE INSERT & Sprawdza minimalny wiek ucznia (5 lat) \\
trg\_lekcja\_konflikt\_ucznia & BEFORE INSERT & Sprawdza czy uczeń nie ma innej lekcji o tej godzinie \\
trg\_lekcja\_konflikt\_nauczyciela & BEFORE INSERT & Sprawdza czy nauczyciel nie ma innej lekcji o tej godzinie \\
trg\_lekcja\_godziny\_dziecka & BEFORE INSERT/UPDATE & Dzieci (<15 lat) mogą mieć lekcje tylko 14:00--19:00 \\
trg\_lekcja\_limit\_nauczyciela & BEFORE INSERT & Sprawdza limit 6h dziennie dla nauczyciela \\
trg\_lekcja\_limit\_ucznia & BEFORE INSERT & Sprawdza limit 2 lekcji dziennie dla ucznia \\
trg\_lekcja\_konflikt\_sali & BEFORE INSERT & Sprawdza czy sala nie jest zajęta \\
trg\_lekcja\_w\_semestrze & BEFORE INSERT & Sprawdza czy data lekcji mieści się w aktywnym semestrze \\
\hline
\end{tabular}
\caption{Triggery walidacyjne}
\end{table}

\subsection{Triggery kontroli semestrów}

\begin{table}[h]
\centering
\begin{tabular}{|l|l|p{5cm}|}
\hline
\textbf{Trigger} & \textbf{Typ} & \textbf{Działanie} \\
\hline
trg\_semestr\_tylko\_jeden\_aktywny & AFTER UPDATE & Dezaktywuje inne semestry przy aktywacji jednego \\
\hline
\end{tabular}
\caption{Triggery kontroli semestrów}
\end{table}

\subsection{Triggery blokujące usuwanie}

\begin{table}[h]
\centering
\begin{tabular}{|l|l|p{5cm}|}
\hline
\textbf{Trigger} & \textbf{Typ} & \textbf{Działanie} \\
\hline
trg\_uczen\_przed\_usunieciem & BEFORE DELETE & Blokuje usunięcie ucznia z zapisami \\
trg\_nauczyciel\_przed\_usunieciem & BEFORE DELETE & Blokuje usunięcie nauczyciela z lekcjami \\
trg\_kurs\_przed\_usunieciem & BEFORE DELETE & Blokuje usunięcie kursu z zapisami \\
trg\_sala\_przed\_usunieciem & BEFORE DELETE & Blokuje usunięcie sali z lekcjami \\
trg\_semestr\_przed\_usunieciem & BEFORE DELETE & Blokuje usunięcie aktywnego semestru \\
\hline
\end{tabular}
\caption{Triggery blokujące usuwanie}
\end{table}

\subsection{Triggery audytowe}

\begin{table}[h]
\centering
\begin{tabular}{|l|l|p{5cm}|}
\hline
\textbf{Trigger} & \textbf{Typ} & \textbf{Działanie} \\
\hline
trg\_ocena\_audit & AFTER INSERT/UPDATE/DELETE & Logowanie operacji na ocenach \\
trg\_lekcja\_status\_audit & AFTER UPDATE & Logowanie zmian statusu lekcji \\
\hline
\end{tabular}
\caption{Triggery audytowe}
\end{table}

\subsection{Szczegółowy opis wybranych triggerów}

\subsubsection{trg\_lekcja\_godziny\_dziecka}
Uczniowie poniżej 15 lat uczęszczają do zwykłej szkoły (podstawowej lub średniej). Ich zajęcia kończą się około 13:00--14:00, dlatego lekcje muzyki mogą odbywać się tylko w godzinach 14:00--19:00.

\begin{verbatim}
IF v_wiek < 15 THEN
    v_godzina := TO_NUMBER(SUBSTR(:NEW.godzina_start, 1, 2));
    IF v_godzina < 14 OR v_godzina >= 19 THEN
        RAISE_APPLICATION_ERROR(-20050, 
            'Dziecko moze miec lekcje tylko 14:00-19:00');
    END IF;
END IF;
\end{verbatim}

\subsubsection{trg\_lekcja\_limit\_nauczyciela}
Nauczyciel może prowadzić maksymalnie 6 godzin (360 minut) lekcji dziennie. Trigger sumuje czas wszystkich zaplanowanych lekcji nauczyciela w danym dniu i blokuje dodanie kolejnej, jeśli limit zostałby przekroczony.

\subsubsection{trg\_semestr\_tylko\_jeden\_aktywny}
W systemie może być aktywny tylko jeden semestr. Po ustawieniu flagi czy\_aktywny='T' dla jednego semestru, trigger automatycznie dezaktywuje wszystkie pozostałe.

\subsection{Kody błędów}
\begin{table}[h]
\centering
\begin{tabular}{|l|p{8cm}|}
\hline
\textbf{Kod} & \textbf{Opis} \\
\hline
-20001 & Uczeń musi mieć minimum 5 lat \\
-20020 & Konflikt nauczyciela (ma już lekcję o tej godzinie) \\
-20021 & Konflikt ucznia (ma już lekcję o tej godzinie) \\
-20030 & Nie można usunąć ucznia z zapisami \\
-20031 & Nie można usunąć nauczyciela z lekcjami \\
-20032 & Nie można usunąć kursu z zapisami \\
-20033 & Nie można usunąć sali z lekcjami \\
-20050 & Dziecko może mieć lekcje tylko 14:00--19:00 \\
-20051 & Nauczyciel przekroczył limit 6h dziennie \\
-20052 & Uczeń przekroczył limit 2 lekcji dziennie \\
-20054 & Sala jest zajęta o tej godzinie \\
-20055 & Lekcja poza zakresem dat aktywnego semestru \\
-20060 & Nie można usunąć aktywnego semestru \\
\hline
\end{tabular}
\caption{Kody błędów aplikacji}
\end{table}

% =============================================================================
% 6. PRZYJĘTE OGRANICZENIA
% =============================================================================
\section{Przyjęte ograniczenia}

\subsection{Ograniczenia dotyczące uczniów}
\begin{enumerate}[nosep]
    \item Minimalny wiek ucznia: 5 lat
    \item Adres email musi być unikalny
    \item Uczeń może mieć maksymalnie 2 lekcje dziennie
    \item Dzieci poniżej 15 lat mogą mieć lekcje tylko w godzinach 14:00--19:00 (uczęszczają do zwykłej szkoły)
    \item Nie można usunąć ucznia, który ma aktywne zapisy na kursy
\end{enumerate}

\subsection{Ograniczenia dotyczące nauczycieli}
\begin{enumerate}[nosep]
    \item Nauczyciel może uczyć maksymalnie 5 instrumentów (VARRAY)
    \item Adres email musi być unikalny
    \item Nauczyciel może prowadzić maksymalnie 6 godzin (360 minut) lekcji dziennie
    \item Nauczyciel nie może mieć dwóch lekcji o tej samej godzinie
    \item Nie można usunąć nauczyciela, który ma przypisane lekcje
\end{enumerate}

\subsection{Ograniczenia dotyczące lekcji}
\begin{enumerate}[nosep]
    \item Lekcje tylko w godzinach 08:00--20:00
    \item Czas trwania lekcji: 30, 45, 60 lub 90 minut
    \item Statusy lekcji: zaplanowana, odbyta, odwołana
    \item Uczeń nie może mieć dwóch lekcji o tej samej godzinie
    \item Nauczyciel nie może mieć dwóch lekcji o tej samej godzinie
    \item Sala nie może być zajęta przez dwie lekcje o tej samej godzinie
    \item Data lekcji musi mieścić się w zakresie dat aktywnego semestru
\end{enumerate}

\subsection{Ograniczenia dotyczące ocen}
\begin{enumerate}[nosep]
    \item Oceny w skali 1--6 (polska skala szkolna)
    \item Obszary oceny: technika, teoria, słuch, rytm, interpretacja, ogólna
\end{enumerate}

\subsection{Ograniczenia dotyczące kursów}
\begin{enumerate}[nosep]
    \item Typy kursów: indywidualny, grupowy
    \item Cena za lekcję musi być większa od 0
    \item Domyślny czas lekcji: 30, 45, 60 lub 90 minut
    \item Nie można usunąć kursu, który ma aktywne zapisy
\end{enumerate}

\subsection{Ograniczenia dotyczące sal}
\begin{enumerate}[nosep]
    \item Nazwa sali musi być unikalna
    \item Pojemność sali musi być większa od 0
    \item Nie można usunąć sali, która ma przypisane lekcje
\end{enumerate}

\subsection{Ograniczenia dotyczące semestrów}
\begin{enumerate}[nosep]
    \item Nazwa semestru musi być unikalna
    \item Data zakończenia musi być późniejsza niż data rozpoczęcia
    \item W systemie może być aktywny tylko jeden semestr
    \item Nie można usunąć aktywnego semestru
\end{enumerate}

\subsection{Ograniczenia dotyczące zapisów}
\begin{enumerate}[nosep]
    \item Statusy zapisu: aktywny, zawieszony, zakończony
\end{enumerate}

% =============================================================================
% 7. ROLE UŻYTKOWNIKÓW
% =============================================================================
\section{Role użytkowników}

\subsection{Administrator (ROLA\_ADMIN\_SZKOLY)}
Pełny dostęp do systemu:
\begin{itemize}[nosep]
    \item Zarządzanie wszystkimi danymi (CRUD na wszystkich tabelach)
    \item Zarządzanie semestrami i salami
    \item Wykonywanie wszystkich pakietów
    \item Zarządzanie użytkownikami i uprawnieniami
\end{itemize}

\subsection{Nauczyciel (ROLA\_NAUCZYCIEL)}
Prowadzenie lekcji i ocenianie:
\begin{itemize}[nosep]
    \item Odczyt danych uczniów, kursów, sal, semestrów
    \item Tworzenie i aktualizacja lekcji
    \item Dodawanie i modyfikacja ocen postępu
    \item Generowanie raportów dziennych
    \item Sprawdzanie dostępności sal
\end{itemize}

\subsection{Sekretariat (ROLA\_SEKRETARIAT)}
Zarządzanie uczniami i zapisami:
\begin{itemize}[nosep]
    \item Pełne zarządzanie danymi uczniów
    \item Pełne zarządzanie zapisami na kursy
    \item Odczyt danych nauczycieli, kursów, lekcji, ocen
    \item Sprawdzanie dostępności sal
    \item Generowanie list i raportów
\end{itemize}

\begin{table}[h]
\centering
\begin{tabular}{|l|c|c|c|}
\hline
\textbf{Tabela} & \textbf{Admin} & \textbf{Nauczyciel} & \textbf{Sekretariat} \\
\hline
t\_uczen & CRUD & R & CRUD \\
t\_nauczyciel & CRUD & R & R \\
t\_kurs & CRUD & R & R \\
t\_zapis & CRUD & R & CRUD \\
t\_lekcja & CRUD & CRU & R \\
t\_ocena\_postepu & CRUD & CRU & R \\
t\_sala & CRUD & R & R \\
t\_semestr & CRUD & R & R \\
\hline
\end{tabular}
\caption{Macierz uprawnień (C=Create, R=Read, U=Update, D=Delete)}
\end{table}

\subsection{Użytkownicy testowi}
\begin{itemize}[nosep]
    \item admin\_szkoly (hasło: Admin123!) -- rola ROLA\_ADMIN\_SZKOLY
    \item nauczyciel\_jan (hasło: Nauczyciel123!) -- rola ROLA\_NAUCZYCIEL
    \item sekretariat\_anna (hasło: Sekretariat123!) -- rola ROLA\_SEKRETARIAT
\end{itemize}

% =============================================================================
% 8. TESTY
% =============================================================================
\section{Testy}

Plik 06\_testy.sql zawiera 46 kompleksowych testów w 7 kategoriach.

\begin{table}[h]
\centering
\begin{tabular}{|l|c|p{6cm}|}
\hline
\textbf{Kategoria} & \textbf{Testów} & \textbf{Zakres} \\
\hline
1. Typy (metody) & 8 & Wszystkie 9 typów obiektowych \\
2. Tabele (CHECK) & 6 & Ograniczenia CHECK na kolumnach \\
3. Pakiety & 13 & Wszystkie 26 procedur/funkcji \\
4. Triggery & 8 & Walidacja i blokady \\
5. Scenariusze & 4 & Cykle życia, planowanie, limity \\
6. Usuwanie & 5 & Blokowanie usuwania powiązanych danych \\
7. Użytkownicy & 2 & Role i uprawnienia \\
\hline
\textbf{RAZEM} & \textbf{46} & \\
\hline
\end{tabular}
\caption{Kategorie testów}
\end{table}

\subsection{Przykładowe testy}
\begin{itemize}[nosep]
    \item Test 1.1: Metody t\_uczen\_obj (wiek, czy\_pelnoletni, pelne\_dane)
    \item Test 2.1: CHECK -- nieprawidłowy status lekcji
    \item Test 3.5: pkg\_uczen.lista\_uczniow
    \item Test 4.2: Trigger -- dziecko nie może mieć lekcji o 10:00
    \item Test 4.4: Trigger -- uczeń max 2 lekcje dziennie
    \item Test 5.2: Scenariusz -- sprawdzenie wszystkich godzin dla dziecka
    \item Test 6.1: Blokada usuwania ucznia z zapisami
\end{itemize}

% =============================================================================
% 9. DIAGRAM RELACJI OBIEKTÓW
% =============================================================================
\section{Diagram relacji obiektów}

\begin{center}
\begin{tikzpicture}[
    node distance=2cm,
    every node/.style={font=\small},
    type/.style={rectangle, draw, fill=blue!20, minimum width=2.8cm, minimum height=0.9cm, align=center},
    varray/.style={rectangle, draw, fill=green!20, minimum width=2.2cm, minimum height=0.7cm, align=center},
    arrow/.style={-{Stealth}, thick}
]

% Nowe typy
\node[type] (semestr) {T\_SEMESTR\\{\scriptsize(daty, aktywny)}};
\node[type, right=2.5cm of semestr] (sala) {T\_SALA\\{\scriptsize(pojemność, wyposażenie)}};

% Typy główne
\node[type, below=1.5cm of semestr] (nauczyciel) {T\_NAUCZYCIEL\\{\scriptsize(imię, staż)}};
\node[type, right=2.5cm of nauczyciel] (lekcja) {T\_LEKCJA\\{\scriptsize(data, godzina)}};
\node[type, below=1.5cm of nauczyciel] (uczen) {T\_UCZEŃ\\{\scriptsize(imię, wiek)}};
\node[type, right=2.5cm of uczen] (zapis) {T\_ZAPIS\\{\scriptsize(data, status)}};
\node[type, below=1.5cm of uczen] (kurs) {T\_KURS\\{\scriptsize(typ, cena)}};
\node[type, right=2.5cm of kurs] (ocena) {T\_OCENA\\{\scriptsize(1-6, obszar)}};

% VARRAY
\node[varray, left=1.2cm of nauczyciel] (varray) {VARRAY\\{\scriptsize instrumenty}};

% Strzałki REF
\draw[arrow] (lekcja) -- node[right, font=\scriptsize] {REF} (nauczyciel);
\draw[arrow] (lekcja) -- node[above, font=\scriptsize] {REF} (sala);
\draw[arrow] (lekcja.south) -- ++(0,-0.3) -| node[near start, above, font=\scriptsize] {REF} (uczen.north east);
\draw[arrow] (lekcja.south) -- ++(0,-0.6) -| node[near start, below, font=\scriptsize] {REF} (kurs.north east);

\draw[arrow] (zapis) -- node[above, font=\scriptsize] {REF} (uczen);
\draw[arrow] (zapis.south) -- ++(0,-0.3) -| node[near start, above, font=\scriptsize] {REF} (kurs);
\draw[arrow] (zapis.north) -- ++(0,0.3) -| node[near start, below, font=\scriptsize] {REF} (nauczyciel);

\draw[arrow] (ocena) -- node[above, font=\scriptsize] {REF} (uczen);
\draw[arrow] (ocena.north) -- ++(0,0.5) -| node[near start, above, font=\scriptsize] {REF} (nauczyciel);

% VARRAY
\draw[arrow, dashed] (nauczyciel) -- (varray);

\end{tikzpicture}
\end{center}

\vspace{0.5cm}

\textbf{Legenda:}
\begin{itemize}[nosep]
    \item Strzałka ciągła -- referencja REF (wskaźnik do obiektu)
    \item Strzałka przerywana -- zawiera VARRAY (kolekcja)
    \item Niebieskie prostokąty -- typy obiektowe / tabele
    \item Zielony prostokąt -- typ kolekcji VARRAY
\end{itemize}

% =============================================================================
% 10. PODSUMOWANIE
% =============================================================================
\section{Podsumowanie}

Projekt obiektowej bazy danych dla szkoły muzycznej spełnia wszystkie wymagania:

\begin{table}[h]
\centering
\begin{tabular}{|p{8cm}|c|}
\hline
\textbf{Wymaganie} & \textbf{Realizacja} \\
\hline
Definicje typów obiektowych z metodami & 9 typów, 15 metod \\
Tabele obiektowe (wierszowe) & 8 tabel \\
Referencja i dereferencja & 8 REF, DEREF \\
Wstawianie danych z referencją & Tak \\
VARRAY do relacji 1:N & t\_lista\_instrumentow \\
Pakiety PL/SQL & 5 pakietów, 26 procedur \\
Kursory i REF kursory & 3 typy kursorów \\
Obsługa błędów & EXCEPTION, RAISE \\
Wyzwalacze & 16 triggerów \\
Role użytkowników & 3 role \\
Testy & 46 testów \\
\hline
\end{tabular}
\caption{Realizacja wymagań projektowych}
\end{table}

\subsection{Pliki projektu}
\begin{enumerate}[nosep]
    \item 01\_typy.sql -- definicje 9 typów obiektowych z 15 metodami
    \item 02\_tabele.sql -- 8 tabel obiektowych, 8 sekwencji, indeksy, CHECK
    \item 03\_pakiety.sql -- 5 pakietów z 26 procedurami i funkcjami
    \item 04\_triggery.sql -- 16 wyzwalaczy
    \item 05\_dane.sql -- dane testowe
    \item 06\_testy.sql -- 46 testów w 7 kategoriach
    \item 07\_uzytkownicy.sql -- 3 role, 3 użytkownicy
\end{enumerate}

\end{document}
