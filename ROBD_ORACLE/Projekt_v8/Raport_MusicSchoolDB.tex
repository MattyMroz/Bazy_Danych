\documentclass[12pt,a4paper]{article}
\usepackage[utf8]{inputenc}
\usepackage[T1]{fontenc}
\usepackage{lmodern}
\usepackage[polish]{babel}
\usepackage{geometry}
\usepackage{graphicx}
\usepackage{array}
\usepackage{booktabs}
\usepackage{enumitem}
\usepackage{amssymb}
\usepackage{tikz}
\usetikzlibrary{shapes.geometric, arrows.meta, positioning}
\usepackage[hidelinks]{hyperref}
\usepackage{longtable}

\geometry{margin=2.5cm}

\begin{document}

% =============================================================================
% STRONA TYTULOWA
% =============================================================================
\begin{titlepage}
    \centering
    \vspace*{1cm}
    
    \Large\textbf{Politechnika Łódzka}\\[0.3cm]
    {\large Wydział Elektrotechniki, Elektroniki, Informatyki i Automatyki}\\[2cm]
    
    {\Huge\textbf{Szkoła Muzyczna I Stopnia}}\\[0.5cm]
    {\Large Obiektowa Baza Danych Oracle}\\[1cm]
    
    {\large Rozproszone i Obiektowe Bazy Danych}\\[3cm]
    
    \begin{tabular}{ll}
        \textbf{Autorzy:} & Igor Typiński (251237) \\
                          & Mateusz Mróz (251190) \\[0.5cm]
        \textbf{Grupa:}   & 5 \\[0.5cm]
        \textbf{Temat:}   & Szkoła muzyczna (z naciskiem na rozwój ucznia)
    \end{tabular}
    
    \vfill
    {\large Łódź, luty 2026}
\end{titlepage}

% =============================================================================
% SPIS TRESCI
% =============================================================================
\tableofcontents
\newpage

% =============================================================================
% 1. OPIS PROJEKTU
% =============================================================================
\section{Opis projektu}

\subsection{Cel i zakres}

Projekt przedstawia obiektową bazę danych dla szkoły muzycznej I stopnia, ze szczególnym uwzględnieniem śledzenia rozwoju uczniów. Szkoła prowadzi nauczanie gry na instrumentach w trybie indywidualnym oraz zajęcia grupowe (kształcenie słuchu, rytmika). System został zaprojektowany z myślą o codziennej pracy sekretariatu, nauczycieli oraz dyrekcji szkoły.

Głównym celem projektu jest stworzenie kompleksowego systemu informatycznego umożliwiającego:

\begin{itemize}
    \item \textbf{Zarządzanie danymi uczniów} -- rejestracja nowych uczniów, przechowywanie danych osobowych, automatyczne obliczanie wieku, przypisanie do grupy (klasy) przez referencję REF
    
    \item \textbf{Zarządzanie danymi nauczycieli} -- ewidencja kadry pedagogicznej, każdy nauczyciel uczy jednego przedmiotu
    
    \item \textbf{Planowanie lekcji} -- tworzenie harmonogramu zajęć indywidualnych (1 uczeń) i grupowych (cała klasa) z walidacją konfliktów terminów oraz heurystyką sugestii wolnego terminu
    
    \item \textbf{Rezerwacja sal lekcyjnych} -- przydzielanie sal z odpowiednim wyposażeniem (przechowywane jako VARRAY) do poszczególnych lekcji
    
    \item \textbf{Ocenianie postępu uczniów} -- wystawianie ocen cząstkowych i semestralnych z walidacją uprawnień nauczyciela
    
    \item \textbf{Kontrola reguł biznesowych} -- automatyczna walidacja uczestników lekcji (albo uczeń albo grupa), zakresu ocen, konfliktów czasowych, kompetencji nauczycieli
    
    \item \textbf{Generowanie raportów} -- plan dnia, plan ucznia, plan nauczyciela, raport grup, statystyki szkoły
\end{itemize}

System uwzględnia specyfikę szkoły muzycznej I stopnia z 6-letnim cyklem nauczania. Lekcje odbywają się w godzinach popołudniowych (14:00--20:00), co odpowiada potrzebom uczniów uczęszczających równolegle do szkoły ogólnokształcącej.

\subsection{Przyjęte ograniczenia}

W projekcie przyjęto następujące ograniczenia biznesowe:

\begin{enumerate}
    \item \textbf{Cykl nauczania:} 6 lat -- klasy I-VI szkoły muzycznej I stopnia
    
    \item \textbf{Godziny pracy szkoły:} 14:00--20:00 -- lekcje mogą być planowane tylko w tych godzinach
    
    \item \textbf{Dni pracy szkoły:} poniedziałek--piątek -- szkoła nie prowadzi zajęć w weekendy
    
    \item \textbf{Czas trwania lekcji:} stały 45 minut dla wszystkich zajęć
    
    \item \textbf{Siatka godzinowa:} lekcje rozpoczynają się o pełnych godzinach (14:00, 15:00, 16:00...)
    
    \item \textbf{Jeden instrument na ucznia:} każdy uczeń uczy się jednego instrumentu głównego
    
    \item \textbf{Jeden przedmiot na nauczyciela:} każdy nauczyciel uczy jednego przedmiotu
    
    \item \textbf{XOR lekcji:} lekcja jest ALBO indywidualna (1 uczeń) ALBO grupowa (klasa) -- nigdy obie opcje jednocześnie
    
    \item \textbf{Maksymalne wyposażenie sali:} 10 elementów (ograniczenie VARRAY)
    
    \item \textbf{Skala ocen:} 1--6 (polska skala szkolna)
    
    \item \textbf{Typy sal:} indywidualna lub grupowa
    
    \item \textbf{Typy przedmiotów:} indywidualny (instrument) lub grupowy (kształcenie słuchu, rytmika)
    
    \item \textbf{Brak konfliktów czasowych:} ta sama sala, nauczyciel, uczeń lub grupa nie mogą mieć dwóch lekcji w tym samym czasie
    
    \item \textbf{Walidacja kompetencji:} nauczyciel może prowadzić tylko lekcje z przedmiotu, który jest do niego przypisany
    
    \item \textbf{Walidacja typu sali:} lekcje grupowe wymagają sali typu 'grupowa'
\end{enumerate}

\subsection{Technologia}

\begin{itemize}
    \item \textbf{Oracle Database} -- obiektowo-relacyjna baza danych
    \item \textbf{Podejście obiektowo-relacyjne} -- typy obiektowe z metodami, REF/DEREF, VARRAY
    \item \textbf{Język PL/SQL} -- pakiety, procedury, funkcje, triggery
\end{itemize}

% =============================================================================
% 2. TYPY OBIEKTOWE
% =============================================================================
\newpage
\section{Typy obiektowe}

W projekcie zdefiniowano 8 typów obiektowych z łącznie 10 metodami.

\begin{table}[h]
\centering
\begin{tabular}{|l|c|p{7cm}|}
\hline
\textbf{Typ} & \textbf{Metody} & \textbf{Opis} \\
\hline
t\_wyposazenie & -- & VARRAY(10) nazw elementów wyposażenia \\
t\_przedmiot & 1 & Przedmiot nauczania (indywidualny/grupowy) \\
t\_grupa & -- & Klasa uczniów (symbol, poziom I-VI) \\
t\_nauczyciel & 2 & Nauczyciel z REF do przedmiotu \\
t\_sala & 2 & Sala lekcyjna z VARRAY wyposażenia \\
t\_uczen & 2 & Uczeń z REF do grupy \\
t\_lekcja & 2 & Lekcja z wyborem: REF do ucznia lub grupy \\
t\_ocena & 1 & Ocena postępu ucznia \\
\hline
\multicolumn{2}{|r|}{\textbf{Razem:}} & \textbf{8 typów, 10 metod} \\
\hline
\end{tabular}
\caption{Typy obiektowe w projekcie}
\end{table}

\subsection{t\_wyposazenie}

Kolekcja VARRAY przechowująca nazwy elementów wyposażenia sali lekcyjnej.

\texttt{CREATE OR REPLACE TYPE t\_wyposazenie AS VARRAY(10) OF VARCHAR2(50);}

Ograniczenie do 10 elementów wynika z założenia, że sala ma skończoną liczbę istotnych elementów wyposażenia (fortepian, metronom, lustro, pulpity itp.). Kolekcja przeszukiwana jest przez funkcję prywatną \texttt{sala\_ma\_instrument()} przy planowaniu lekcji.

\subsection{t\_przedmiot}

Reprezentuje przedmiot nauczania w słowniku przedmiotów.

\textbf{Atrybuty:}
\begin{itemize}[nosep]
    \item id (NUMBER) -- unikalny identyfikator
    \item nazwa (VARCHAR2(50)) -- nazwa przedmiotu (np. Fortepian, Kształcenie słuchu)
    \item typ (VARCHAR2(20)) -- typ: 'indywidualny' lub 'grupowy'
    \item czas\_min (NUMBER) -- czas trwania lekcji (stałe 45 minut)
\end{itemize}

\textbf{Metody:}
\begin{itemize}[nosep]
    \item czy\_grupowy() RETURN VARCHAR2 -- zwraca 'T' jeśli typ = 'grupowy', inaczej 'N'
\end{itemize}

\subsection{t\_grupa}

Reprezentuje klasę (grupę) uczniów w szkole muzycznej.

\textbf{Atrybuty:}
\begin{itemize}[nosep]
    \item id (NUMBER) -- unikalny identyfikator
    \item symbol (VARCHAR2(10)) -- symbol grupy (np. '1A', '2A', '3A')
    \item poziom (NUMBER) -- poziom klasy 1-6 (klasa I-VI szkoły muzycznej)
\end{itemize}

Typ nie posiada metod -- jest prostym kontenerem danych słownikowych.

\subsection{t\_nauczyciel}

Reprezentuje nauczyciela szkoły muzycznej wraz z referencją do przedmiotu.

\textbf{Atrybuty:}
\begin{itemize}[nosep]
    \item id (NUMBER) -- unikalny identyfikator
    \item imie (VARCHAR2(50)) -- imię nauczyciela
    \item nazwisko (VARCHAR2(50)) -- nazwisko nauczyciela
    \item data\_zatr (DATE) -- data zatrudnienia
    \item ref\_przedmiot (REF t\_przedmiot) -- referencja do przedmiotu który uczy
\end{itemize}

\textbf{Metody:}
\begin{itemize}[nosep]
    \item pelne\_nazwisko() RETURN VARCHAR2 -- zwraca imię i nazwisko
    \item staz\_lat() RETURN NUMBER -- oblicza liczbę lat pracy w szkole
\end{itemize}

\subsection{t\_sala}

Reprezentuje salę lekcyjną z kolekcją VARRAY wyposażenia.

\textbf{Atrybuty:}
\begin{itemize}[nosep]
    \item id (NUMBER) -- unikalny identyfikator
    \item numer (VARCHAR2(10)) -- numer sali (np. '101', '102')
    \item typ (VARCHAR2(20)) -- typ: 'indywidualna' lub 'grupowa'
    \item pojemnosc (NUMBER) -- maksymalna liczba osób
    \item wyposazenie (t\_wyposazenie) -- VARRAY elementów wyposażenia
\end{itemize}

\textbf{Metody:}
\begin{itemize}[nosep]
    \item czy\_grupowa() RETURN VARCHAR2 -- zwraca 'T' jeśli typ = 'grupowa', inaczej 'N'
    \item lista\_wyposazenia() RETURN VARCHAR2 -- zwraca wyposażenie jako tekst rozdzielony przecinkami (iteracja po VARRAY)
\end{itemize}

\subsection{t\_uczen}

Reprezentuje ucznia szkoły muzycznej z referencją do grupy.

\textbf{Atrybuty:}
\begin{itemize}[nosep]
    \item id (NUMBER) -- unikalny identyfikator
    \item imie (VARCHAR2(50)) -- imię ucznia
    \item nazwisko (VARCHAR2(50)) -- nazwisko ucznia
    \item data\_ur (DATE) -- data urodzenia
    \item instrument (VARCHAR2(50)) -- instrument główny (np. Fortepian, Skrzypce)
    \item ref\_grupa (REF t\_grupa) -- referencja do grupy (klasy)
\end{itemize}

\textbf{Metody:}
\begin{itemize}[nosep]
    \item pelne\_nazwisko() RETURN VARCHAR2 -- zwraca imię i nazwisko
    \item wiek() RETURN NUMBER -- oblicza aktualny wiek w latach
\end{itemize}

\subsection{t\_lekcja}

Reprezentuje pojedynczą lekcję muzyki z wyborem uczestnika (albo uczeń albo grupa).

\textbf{Atrybuty:}
\begin{itemize}[nosep]
    \item id (NUMBER) -- unikalny identyfikator
    \item ref\_przedmiot (REF t\_przedmiot) -- referencja do przedmiotu
    \item ref\_nauczyciel (REF t\_nauczyciel) -- referencja do nauczyciela
    \item ref\_sala (REF t\_sala) -- referencja do sali
    \item ref\_uczen (REF t\_uczen) -- referencja do ucznia (lekcja indywidualna)
    \item ref\_grupa (REF t\_grupa) -- referencja do grupy (lekcja grupowa)
    \item data\_lekcji (DATE) -- data lekcji
    \item godz\_rozp (NUMBER) -- godzina rozpoczęcia (14, 15, 16... -- pełne godziny)
    \item czas\_min (NUMBER) -- czas trwania w minutach (stałe 45)
\end{itemize}

\textbf{Metody:}
\begin{itemize}[nosep]
    \item godzina\_koniec() RETURN NUMBER -- oblicza godzinę zakończenia (np. 14.75 dla lekcji 14:00-14:45)
    \item czy\_indywidualna() RETURN VARCHAR2 -- zwraca 'T' jeśli ref\_uczen IS NOT NULL, inaczej 'N'
\end{itemize}

\textbf{Reguła przypisania:} Lekcja musi mieć wypełnione ALBO ref\_uczen (lekcja indywidualna) ALBO ref\_grupa (lekcja grupowa), nigdy oba jednocześnie i nigdy żadne. Walidowane przez trigger \texttt{trg\_lekcja\_xor}.

\subsection{t\_ocena}

Reprezentuje ocenę postępu ucznia.

\textbf{Atrybuty:}
\begin{itemize}[nosep]
    \item id (NUMBER) -- unikalny identyfikator
    \item ref\_uczen (REF t\_uczen) -- referencja do ucznia
    \item ref\_nauczyciel (REF t\_nauczyciel) -- referencja do nauczyciela wystawiającego
    \item ref\_przedmiot (REF t\_przedmiot) -- referencja do przedmiotu
    \item wartosc (NUMBER) -- wartość oceny 1--6
    \item data\_oceny (DATE) -- data wystawienia oceny
    \item semestralna (VARCHAR2(1)) -- 'T' dla oceny semestralnej, 'N' dla cząstkowej
\end{itemize}

\textbf{Metody:}
\begin{itemize}[nosep]
    \item opis\_oceny() RETURN VARCHAR2 -- zwraca ocenę słownie (celujący, bardzo dobry, dobry, dostateczny, dopuszczający, niedostateczny)
\end{itemize}

% =============================================================================
% 3. TABELE OBIEKTOWE
% =============================================================================
\newpage
\section{Tabele obiektowe}

Utworzono 7 tabel obiektowych przechowujących dane.

\begin{table}[h]
\centering
\begin{tabular}{|l|l|l|}
\hline
\textbf{Tabela} & \textbf{Typ bazowy} & \textbf{Referencje / Kolekcje} \\
\hline
przedmioty & t\_przedmiot & -- \\
grupy & t\_grupa & -- \\
nauczyciele & t\_nauczyciel & ref\_przedmiot \\
sale & t\_sala & zawiera VARRAY (wyposazenie) \\
uczniowie & t\_uczen & ref\_grupa \\
lekcje & t\_lekcja & ref\_przedmiot, ref\_nauczyciel, ref\_sala, \\
       &           & ref\_uczen (indyw.), ref\_grupa (grup.) \\
oceny & t\_ocena & ref\_uczen, ref\_nauczyciel, ref\_przedmiot \\
\hline
\end{tabular}
\caption{Tabele obiektowe i ich referencje}
\end{table}

\subsection{przedmioty}

Słownik przedmiotów nauczania.

\begin{itemize}[nosep]
    \item Klucz główny: id
    \item Ograniczenie UNIQUE: nazwa
    \item Ograniczenia NOT NULL: nazwa, typ, czas\_min
    \item CHECK: typ IN ('indywidualny', 'grupowy')
    \item CHECK: czas\_min = 45
\end{itemize}

\subsection{grupy}

Słownik grup (klas) uczniów.

\begin{itemize}[nosep]
    \item Klucz główny: id
    \item Ograniczenie UNIQUE: symbol
    \item Ograniczenia NOT NULL: symbol, poziom
    \item CHECK: poziom BETWEEN 1 AND 6
\end{itemize}

\subsection{nauczyciele}

Dane nauczycieli z referencją do przedmiotu.

\begin{itemize}[nosep]
    \item Klucz główny: id
    \item Ograniczenia NOT NULL: imie, nazwisko, data\_zatr
    \item Referencja: ref\_przedmiot → przedmioty
\end{itemize}

\subsection{sale}

Informacje o salach lekcyjnych z kolekcją VARRAY wyposażenia.

\begin{itemize}[nosep]
    \item Klucz główny: id
    \item Ograniczenie UNIQUE: numer
    \item Ograniczenia NOT NULL: numer, typ, pojemnosc
    \item CHECK: typ IN ('indywidualna', 'grupowa')
    \item CHECK: pojemnosc > 0
    \item Zawiera kolekcję t\_wyposazenie (VARRAY)
\end{itemize}

\subsection{uczniowie}

Dane uczniów szkoły muzycznej z referencją do grupy.

\begin{itemize}[nosep]
    \item Klucz główny: id
    \item Ograniczenia NOT NULL: imie, nazwisko, data\_ur, instrument
    \item Referencja: ref\_grupa → grupy
\end{itemize}

\subsection{lekcje}

Zaplanowane lekcje muzyki z wyborem typu uczestnictwa.

\begin{itemize}[nosep]
    \item Klucz główny: id
    \item Ograniczenia NOT NULL: data\_lekcji, godz\_rozp, czas\_min
    \item CHECK: godz\_rozp BETWEEN 14 AND 19
    \item CHECK: czas\_min = 45
    \item Referencje: ref\_przedmiot, ref\_nauczyciel, ref\_sala
    \item Relacja uczestników: ref\_uczen (dla indywidualnych), ref\_grupa (dla grupowych)
    \item Trigger: trg\_lekcja\_xor wymusza poprawność danych
\end{itemize}

\subsection{oceny}

Oceny postępu uczniów.

\begin{itemize}[nosep]
    \item Klucz główny: id
    \item Ograniczenia NOT NULL: wartosc, data\_oceny, semestralna
    \item CHECK: wartosc BETWEEN 1 AND 6
    \item CHECK: semestralna IN ('T', 'N')
    \item Referencje: ref\_uczen, ref\_nauczyciel, ref\_przedmiot
\end{itemize}

\subsection{Referencje (REF/DEREF)}

W projekcie zastosowano 9 referencji do modelowania relacji między obiektami:

\begin{itemize}[nosep]
    \item Nauczyciel wskazuje na przedmiot, którego uczy (1 REF)
    \item Uczeń wskazuje na grupę, do której należy (1 REF)
    \item Lekcja wskazuje na przedmiot, nauczyciela, salę oraz ucznia lub grupę (4-5 REF)
    \item Ocena wskazuje na ucznia, nauczyciela i przedmiot (3 REF)
\end{itemize}

Dzięki DEREF możliwe jest odwołanie się do atrybutów i metod obiektu wskazywanego:

\texttt{SELECT DEREF(l.ref\_uczen).pelne\_nazwisko() AS uczen,}

\texttt{\hspace{1.5cm}DEREF(l.ref\_sala).numer AS sala,}

\texttt{\hspace{1.5cm}DEREF(l.ref\_przedmiot).nazwa AS przedmiot}

\texttt{FROM lekcje l WHERE l.ref\_uczen IS NOT NULL;}

\subsection{Sekwencje}

Utworzono 7 sekwencji do generowania identyfikatorów:

\begin{itemize}[nosep]
    \item seq\_przedmioty -- dla tabeli przedmioty
    \item seq\_grupy -- dla tabeli grupy
    \item seq\_nauczyciele -- dla tabeli nauczyciele
    \item seq\_sale -- dla tabeli sale
    \item seq\_uczniowie -- dla tabeli uczniowie
    \item seq\_lekcje -- dla tabeli lekcje
    \item seq\_oceny -- dla tabeli oceny
\end{itemize}

% =============================================================================
% 4. PAKIETY PL/SQL
% =============================================================================
\newpage
\section{Pakiety PL/SQL}

Logika biznesowa zaimplementowana w 5 pakietach z łącznie 25 podprogramami.

\begin{table}[h]
\centering
\begin{tabular}{|l|c|p{7cm}|}
\hline
\textbf{Pakiet} & \textbf{Podprogramy} & \textbf{Funkcjonalności} \\
\hline
pkg\_slowniki & 9 & Zarządzanie słownikami (przedmioty, grupy, sale) \\
pkg\_osoby & 7 & Zarządzanie nauczycielami i uczniami \\
pkg\_lekcje & 5 (+3 prywatne) & Planowanie lekcji z walidacją i heurystyką \\
pkg\_oceny & 4 (+1 prywatna) & Zarządzanie ocenami \\
pkg\_raporty & 2 & Raporty i statystyki \\
\hline
\multicolumn{2}{|r|}{\textbf{Razem:}} & \textbf{25 podprogramów publicznych} \\
\hline
\end{tabular}
\caption{Pakiety PL/SQL}
\end{table}

\subsection{pkg\_slowniki}

Pakiet do zarządzania słownikami (przedmioty, grupy, sale).

\textbf{Procedury dodawania:}
\begin{itemize}[nosep]
    \item dodaj\_przedmiot(p\_nazwa, p\_typ) -- dodaje przedmiot do słownika
    \item dodaj\_grupe(p\_symbol, p\_poziom) -- dodaje grupę (klasę)
    \item dodaj\_sale(p\_numer, p\_typ, p\_pojemnosc, p\_wyposazenie) -- dodaje salę z VARRAY wyposażenia
\end{itemize}

\textbf{Funkcje pobierania referencji:}
\begin{itemize}[nosep]
    \item get\_ref\_przedmiot(p\_id) RETURN REF t\_przedmiot
    \item get\_ref\_grupa(p\_id) RETURN REF t\_grupa
    \item get\_ref\_sala(p\_id) RETURN REF t\_sala
\end{itemize}

\textbf{Procedury listowania:}
\begin{itemize}[nosep]
    \item lista\_przedmiotow() -- wyświetla listę przedmiotów
    \item lista\_grup() -- wyświetla listę grup
    \item lista\_sal() -- wyświetla listę sal z wyposażeniem (wywołuje metodę lista\_wyposazenia())
\end{itemize}

\subsection{pkg\_osoby}

Pakiet do zarządzania nauczycielami i uczniami.

\textbf{Procedury dodawania:}
\begin{itemize}[nosep]
    \item dodaj\_nauczyciela(p\_imie, p\_nazwisko, p\_id\_przedmiotu) -- dodaje nauczyciela z REF do przedmiotu
    \item dodaj\_ucznia(p\_imie, p\_nazwisko, p\_data\_ur, p\_instrument, p\_id\_grupy) -- dodaje ucznia z REF do grupy
\end{itemize}

\textbf{Funkcje pobierania referencji:}
\begin{itemize}[nosep]
    \item get\_ref\_nauczyciel(p\_id) RETURN REF t\_nauczyciel
    \item get\_ref\_uczen(p\_id) RETURN REF t\_uczen
\end{itemize}

\textbf{Procedury listowania:}
\begin{itemize}[nosep]
    \item lista\_nauczycieli() -- wyświetla nauczycieli z przedmiotami (DEREF)
    \item lista\_uczniow() -- wyświetla uczniów z grupami (DEREF)
    \item lista\_uczniow\_grupy(p\_id\_grupy) -- wyświetla uczniów danej grupy (\textbf{kursor jawny} OPEN/FETCH/CLOSE)
\end{itemize}

\subsection{pkg\_lekcje}

Pakiet do zarządzania lekcjami z pełną walidacją konfliktów i heurystyką sugestii terminu.

\textbf{Procedury publiczne:}
\begin{itemize}[nosep]
    \item dodaj\_lekcje\_indywidualna(...) -- planuje lekcję indywidualną z walidacją
    \item dodaj\_lekcje\_grupowa(...) -- planuje lekcję grupową z walidacją
    \item plan\_ucznia(p\_id\_ucznia) -- wyświetla plan ucznia (UNION lekcji indywidualnych i grupowych)
    \item plan\_nauczyciela(p\_id\_nauczyciela) -- wyświetla plan nauczyciela
    \item plan\_dnia(p\_data) -- wyświetla wszystkie lekcje w danym dniu
\end{itemize}

\textbf{Funkcje prywatne:}
\begin{itemize}[nosep]
    \item sprawdz\_kolizje(...) -- sprawdza dostępność sali, nauczyciela, ucznia/grupy
    \item sala\_ma\_instrument(p\_id\_sali, p\_instrument) -- przeszukuje \textbf{VARRAY} wyposażenia sali
    \item znajdz\_alternatywe(...) -- \textbf{heurystyka First Fit} szukająca wolnego terminu
\end{itemize}

\textbf{Walidacje w procedurze dodaj\_lekcje\_indywidualna:}
\begin{itemize}[nosep]
    \item Kompetencje nauczyciela (czy uczy tego przedmiotu -- sprawdzenie REF)
    \item Zgodność instrumentu ucznia z przedmiotem
    \item Konflikt sali -- brak nakładających się terminów
    \item Konflikt nauczyciela -- brak nakładających się terminów
    \item Konflikt ucznia -- brak nakładających się terminów
\end{itemize}

\textbf{Dodatkowe walidacje w procedurze dodaj\_lekcje\_grupowa:}
\begin{itemize}[nosep]
    \item Typ sali -- lekcja grupowa wymaga sali typu 'grupowa'
    \item Przepełnienie sali -- pojemność >= liczba uczniów w grupie
    \item Konflikt grupy -- brak nakładających się terminów
\end{itemize}

\subsection{pkg\_oceny}

Pakiet do zarządzania ocenami postępu uczniów.

\textbf{Procedury publiczne:}
\begin{itemize}[nosep]
    \item wystaw\_ocene(p\_id\_ucznia, p\_id\_nauczyciela, p\_id\_przedmiotu, p\_wartosc) -- ocena cząstkowa
    \item wystaw\_ocene\_semestralna(...) -- ocena semestralna (semestralna='T')
    \item oceny\_ucznia(p\_id\_ucznia) -- wyświetla wszystkie oceny ucznia z opisem słownym
\end{itemize}

\textbf{Funkcja publiczna:}
\begin{itemize}[nosep]
    \item srednia\_ucznia(p\_id\_ucznia, p\_id\_przedmiotu) RETURN NUMBER -- średnia ocen cząstkowych (0 gdy brak)
\end{itemize}

\textbf{Procedura prywatna:}
\begin{itemize}[nosep]
    \item sprawdz\_uprawnienia\_oceniania(...) -- waliduje czy nauczyciel uczy danego przedmiotu
\end{itemize}

\subsection{pkg\_raporty}

Pakiet do generowania raportów i statystyk.

\textbf{Procedury:}
\begin{itemize}[nosep]
    \item raport\_grup() -- wyświetla liczbę uczniów w każdej grupie (podzapytanie skorelowane)
    \item statystyki() -- wyświetla ogólne statystyki szkoły (liczba uczniów, nauczycieli, grup, sal, lekcji, ocen)
\end{itemize}

% =============================================================================
% 5. WYZWALACZE
% =============================================================================
\newpage
\section{Wyzwalacze (Triggery)}

Zdefiniowano 2 wyzwalacze realizujące krytyczne reguły biznesowe. Walidacja konfliktów terminów jest zaimplementowana w pakiecie \texttt{pkg\_lekcje}, aby uniknąć błędu ORA-04091 (Mutating Table).

\begin{table}[h]
\centering
\small
\begin{tabular}{|l|l|p{6cm}|}
\hline
\textbf{Trigger} & \textbf{Typ} & \textbf{Działanie} \\
\hline
trg\_lekcja\_xor & BEFORE INSERT/UPDATE & Weryfikacja uczestników: lekcja musi mieć ALBO ucznia ALBO grupę (nigdy oba, nigdy żadne) \\
\hline
trg\_ocena\_zakres & BEFORE INSERT/UPDATE & Walidacja zakresu ocen 1-6 z przyjaznym komunikatem błędu \\
\hline
\end{tabular}
\caption{Wyzwalacze w projekcie}
\end{table}

\subsection{trg\_lekcja\_xor}

Trigger wymuszający poprawność przypisania uczestników dla lekcji:

\begin{verbatim}
CREATE OR REPLACE TRIGGER trg_lekcja_xor
BEFORE INSERT OR UPDATE ON lekcje
FOR EACH ROW
BEGIN
    IF (:NEW.ref_uczen IS NULL AND :NEW.ref_grupa IS NULL) OR
       (:NEW.ref_uczen IS NOT NULL AND :NEW.ref_grupa IS NOT NULL) THEN
        RAISE_APPLICATION_ERROR(-20001, 
            'Lekcja musi miec ALBO ucznia (indywidualna) ALBO grupe (grupowa)');
    END IF;
END;
\end{verbatim}

\subsection{trg\_ocena\_zakres}

Trigger walidujący zakres ocen z przyjaznym komunikatem:

\begin{verbatim}
CREATE OR REPLACE TRIGGER trg_ocena_zakres
BEFORE INSERT OR UPDATE ON oceny
FOR EACH ROW
BEGIN
    IF :NEW.wartosc < 1 OR :NEW.wartosc > 6 THEN
        RAISE_APPLICATION_ERROR(-20002, 
            'Ocena musi byc w zakresie 1-6. Podano: ' || :NEW.wartosc);
    END IF;
END;
\end{verbatim}

% =============================================================================
% 6. OBSLUGA BLEDOW
% =============================================================================
\newpage
\section{Obsługa błędów}

W projekcie zastosowano mechanizmy obsługi wyjątków z własnymi kodami błędów.

\subsection{Kody błędów aplikacji}

\begin{table}[h]
\centering
\small
\begin{tabular}{|c|l|p{6cm}|}
\hline
\textbf{Kod} & \textbf{Źródło} & \textbf{Znaczenie} \\
\hline
-20001 & trg\_lekcja\_xor & Nieprawidłowe przypisanie uczestników (lekcja musi mieć ucznia LUB grupę) \\
\hline
-20002 & trg\_ocena\_zakres & Ocena poza zakresem 1-6 \\
\hline
-20010 & pkg\_slowniki.get\_ref\_przedmiot & Nie znaleziono przedmiotu o podanym ID \\
\hline
-20011 & pkg\_slowniki.get\_ref\_grupa & Nie znaleziono grupy o podanym ID \\
\hline
-20012 & pkg\_slowniki.get\_ref\_sala & Nie znaleziono sali o podanym ID \\
\hline
-20013 & pkg\_osoby.get\_ref\_nauczyciel & Nie znaleziono nauczyciela o podanym ID \\
\hline
-20014 & pkg\_osoby.get\_ref\_uczen & Nie znaleziono ucznia o podanym ID \\
\hline
-20020 & pkg\_lekcje.dodaj\_lekcje\_indywidualna & Konflikt terminów (+ sugestia alternatywy) \\
\hline
-20021 & pkg\_lekcje.dodaj\_lekcje\_grupowa & Konflikt terminów (+ sugestia alternatywy) \\
\hline
-20030 & pkg\_lekcje & Nauczyciel nie uczy podanego przedmiotu \\
\hline
-20031 & pkg\_lekcje & Lekcja grupowa w sali indywidualnej \\
\hline
-20032 & pkg\_lekcje & Instrument ucznia niezgodny z przedmiotem \\
\hline
-20033 & pkg\_oceny & Nauczyciel nie ma uprawnień do oceniania \\
\hline
-20035 & pkg\_lekcje & Przepełnienie sali (grupa > pojemność) \\
\hline
\end{tabular}
\caption{Kody błędów aplikacji}
\end{table}

\subsection{Przykład komunikatu z sugestią}

Gdy system wykryje konflikt terminów, automatycznie sugeruje alternatywny termin:

\begin{verbatim}
ORA-20020: Blad planowania: Sala jest juz zajeta w tym terminie!
SUGEROWANY TERMIN: 2025-06-02 o godzinie 15:00 w sali 101
\end{verbatim}

% =============================================================================
% 7. HEURYSTYKA SUGESTII TERMINU
% =============================================================================
\newpage
\section{Heurystyka sugestii terminu (First Fit)}

System implementuje algorytm \textbf{First Fit} do automatycznego sugerowania wolnego terminu w przypadku konfliktu.

\subsection{Algorytm znajdz\_alternatywe()}

\begin{enumerate}
    \item Zacznij od następnej godziny po nieudanym terminie
    \item Sprawdzaj godziny robocze (14:00 -- 19:00)
    \item Jeśli dzień się skończy, przeskocz do następnego dnia na 14:00
    \item Szukaj maksymalnie przez 7 dni (limit bezpieczeństwa)
    \item Dla każdego terminu iteruj po dostępnych salach
\end{enumerate}

\subsection{Dopasowanie sali}

\begin{table}[h]
\centering
\begin{tabular}{|l|p{9cm}|}
\hline
\textbf{Typ lekcji} & \textbf{Kryteria doboru sali} \\
\hline
Indywidualna & Funkcja \texttt{sala\_ma\_instrument()} przeszukuje \textbf{VARRAY wyposażenia} sali, szukając elementu pasującego do instrumentu ucznia. Obsługuje synonimy (np. Pianino = Fortepian). \\
\hline
Grupowa & Szuka tylko sal typu 'grupowa' z pojemnością >= liczba uczniów w grupie \\
\hline
\end{tabular}
\caption{Kryteria doboru sali w heurystyce}
\end{table}

\subsection{Funkcja sala\_ma\_instrument()}

Funkcja przeszukuje kolekcję VARRAY wyposażenia sali:

\begin{verbatim}
FUNCTION sala_ma_instrument(p_id_sali NUMBER, p_instrument VARCHAR2) 
RETURN BOOLEAN IS
    v_wyposazenie t_wyposazenie;
BEGIN
    SELECT s.wyposazenie INTO v_wyposazenie FROM sale s WHERE s.id = p_id_sali;
    IF v_wyposazenie IS NULL THEN RETURN FALSE; END IF;
    
    FOR i IN 1..v_wyposazenie.COUNT LOOP
        IF UPPER(v_wyposazenie(i)) LIKE '%' || UPPER(p_instrument) || '%' THEN
            RETURN TRUE;
        END IF;
        -- Synonim: Pianino = Fortepian
        IF UPPER(p_instrument) = 'FORTEPIAN' AND 
           UPPER(v_wyposazenie(i)) LIKE '%PIANINO%' THEN
            RETURN TRUE;
        END IF;
    END LOOP;
    RETURN FALSE;
END;
\end{verbatim}

% =============================================================================
% 8. SCENARIUSZE UZYCIA
% =============================================================================
\newpage
\section{Scenariusze użycia}

Projekt zawiera kompleksowe scenariusze testowe demonstrujące funkcjonalności systemu.

\subsection{Scenariusz 1: Administrator rozszerza ofertę szkoły}

Administrator dodaje nowy przedmiot (Flet), salę z wyposażeniem (VARRAY) oraz nauczyciela:

\begin{itemize}[nosep]
    \item \texttt{pkg\_slowniki.dodaj\_przedmiot('Flet', 'indywidualny')}
    \item \texttt{pkg\_slowniki.dodaj\_sale('105', 'indywidualna', 4, t\_wyposazenie('Flet poprzeczny', 'Pulpit', 'Metronom'))}
    \item \texttt{pkg\_osoby.dodaj\_nauczyciela('Tomasz', 'Flecista', 6)} -- REF do przedmiotu ID=6
\end{itemize}

\subsection{Scenariusz 2: Sekretariat tworzy grupę i zapisuje uczniów}

Sekretariat tworzy nową klasę 4A i zapisuje uczniów z referencją do grupy:

\begin{itemize}[nosep]
    \item \texttt{pkg\_slowniki.dodaj\_grupe('4A', 4)}
    \item \texttt{pkg\_osoby.dodaj\_ucznia('Jakub', 'Melodyjny', DATE '2014-07-20', 'Flet', 4)} -- REF do grupy ID=4
    \item \texttt{pkg\_osoby.lista\_uczniow\_grupy(4)} -- używa \textbf{kursora jawnego}
\end{itemize}

\subsection{Scenariusz 3: Planowanie lekcji grupowych}

Sekretariat planuje zajęcia grupowe z walidacją typu sali i pojemności:

\begin{itemize}[nosep]
    \item \texttt{pkg\_lekcje.dodaj\_lekcje\_grupowa(4, 4, 3, 4, DATE '2025-06-09', 17)}
    \item System waliduje: typ sali = 'grupowa', pojemność >= liczba uczniów, brak konfliktów
\end{itemize}

\subsection{Scenariusz 4: Planowanie lekcji indywidualnych z konfliktem}

Demonstracja walidacji konfliktów i heurystyki sugestii:

\begin{itemize}[nosep]
    \item \textbf{Sukces:} \texttt{pkg\_lekcje.dodaj\_lekcje\_indywidualna(6, 6, 5, 10, DATE '2025-06-09', 14)}
    \item \textbf{Konflikt:} Próba dodania lekcji w zajętym terminie → błąd -20020 z sugestią alternatywnego terminu
    \item System przeszukuje VARRAY wyposażenia sal szukając pasującego instrumentu
\end{itemize}

\subsection{Scenariusz 5: Nauczyciel wystawia oceny}

Nauczyciel wystawia oceny z walidacją uprawnień:

\begin{itemize}[nosep]
    \item \texttt{pkg\_oceny.wystaw\_ocene(10, 6, 6, 5)} -- ocena cząstkowa
    \item \texttt{pkg\_oceny.wystaw\_ocene\_semestralna(10, 6, 6, 5)} -- ocena semestralna
    \item \textbf{Błąd -20033:} Nauczyciel próbuje wystawić ocenę z przedmiotu którego nie uczy
\end{itemize}

\subsection{Scenariusz 6: Uczeń sprawdza oceny i plan}

Uczeń (lub rodzic) przegląda swoje dane:

\begin{itemize}[nosep]
    \item \texttt{pkg\_oceny.oceny\_ucznia(10)} -- lista ocen z opisem słownym (metoda opis\_oceny())
    \item \texttt{SELECT pkg\_oceny.srednia\_ucznia(10, 6) FROM DUAL} -- średnia z przedmiotu
    \item \texttt{pkg\_lekcje.plan\_ucznia(10)} -- plan lekcji (UNION indywidualnych i grupowych)
\end{itemize}

\subsection{Scenariusz 7: Raporty dla dyrekcji}

Dyrekcja generuje raporty podsumowujące:

\begin{itemize}[nosep]
    \item \texttt{pkg\_raporty.raport\_grup()} -- liczba uczniów w każdej klasie
    \item \texttt{pkg\_raporty.statystyki()} -- ogólne statystyki szkoły
    \item \texttt{pkg\_lekcje.plan\_nauczyciela(6)} -- plan nauczyciela (kontrola obciążenia)
\end{itemize}

\subsection{Scenariusz 8: Walidacje i przypadki brzegowe}

Demonstracja walidacji i obsługi błędów:

\begin{itemize}[nosep]
    \item \textbf{Logika przypisania:} Lekcja bez ucznia i bez grupy → błąd -20001
    \item \textbf{Trigger ocen:} Ocena = 7 → błąd -20002
    \item \textbf{Typ sali:} Lekcja grupowa w sali indywidualnej → błąd -20031
    \item \textbf{Instrument:} Uczeń z Fletem na lekcji Fortepianu → błąd -20032
    \item \textbf{Kompetencje:} Nauczyciel Fletu prowadzi lekcję Fortepianu → błąd -20030
\end{itemize}

\subsection{Scenariusz 9: Demonstracja VARRAY}

Pokazanie działania kolekcji VARRAY wyposażenia sal:

\begin{itemize}[nosep]
    \item \texttt{pkg\_slowniki.lista\_sal()} -- lista sal z wyposażeniem
    \item Metoda \texttt{lista\_wyposazenia()} iteruje po VARRAY i zwraca tekst
    \item Funkcja \texttt{sala\_ma\_instrument()} przeszukuje VARRAY przy planowaniu
\end{itemize}

\subsection{Scenariusz 10: Demonstracja REF/DEREF}

Pokazanie działania referencji obiektowych:

\begin{itemize}[nosep]
    \item Nauczyciele z przedmiotami (DEREF na REF do przedmiotu)
    \item Uczniowie z grupami (DEREF na REF do grupy)
    \item Lekcje z pełnymi danymi (wielokrotny DEREF)
\end{itemize}

\subsection{Scenariusz 11: Demonstracja metod obiektowych}

Pokazanie działania metod zdefiniowanych w typach:

\begin{itemize}[nosep]
    \item \texttt{t\_uczen.pelne\_nazwisko()}, \texttt{t\_uczen.wiek()}
    \item \texttt{t\_nauczyciel.pelne\_nazwisko()}, \texttt{t\_nauczyciel.staz\_lat()}
    \item \texttt{t\_przedmiot.czy\_grupowy()}, \texttt{t\_sala.czy\_grupowa()}
    \item \texttt{t\_lekcja.czy\_indywidualna()}, \texttt{t\_lekcja.godzina\_koniec()}
    \item \texttt{t\_ocena.opis\_oceny()}
\end{itemize}

% =============================================================================
% 9. DIAGRAM RELACJI OBIEKTOW
% =============================================================================
\newpage
\section{Diagram relacji obiektów}

\begin{figure}[h]
\centering
\includegraphics[width=\textwidth,keepaspectratio]{Relational_1.png}
\caption{Diagram relacji obiektów w bazie danych}
\end{figure}

% =============================================================================
% 10. STRUKTURA PLIKÓW
% =============================================================================
\newpage
\section{Struktura plików projektu}

\begin{table}[h]
\centering
\begin{tabular}{|l|p{9cm}|}
\hline
\textbf{Plik} & \textbf{Zawartość} \\
\hline
01\_typy.sql & Definicje 8 typów obiektowych z 10 metodami, VARRAY \\
02\_tabele.sql & 7 tabel obiektowych, 7 sekwencji \\
03\_pakiety.sql & 5 pakietów PL/SQL z 25 podprogramami publicznymi \\
04\_triggery.sql & 2 wyzwalacze walidacyjne (typ lekcji, zakres ocen) \\
05\_dane.sql & Dane testowe (5 przedm., 3 grupy, 5 naucz., 4 sale, 9 uczn.) \\
06\_testy.sql & Scenariusze testowe 1-11 (dokumentacja API + testy) \\
07\_testy\_rozszerzone.sql & Scenariusze testowe 12-23 (przypadki brzegowe) \\
\hline
\end{tabular}
\caption{Pliki projektu}
\end{table}

\textbf{Kolejność uruchamiania:}
\begin{enumerate}[nosep]
    \item 01\_typy.sql -- typy obiektowe z metodami
    \item 02\_tabele.sql -- tabele obiektowe i sekwencje
    \item 03\_pakiety.sql -- pakiety PL/SQL
    \item 04\_triggery.sql -- wyzwalacze
    \item 05\_dane.sql -- dane testowe
    \item 06\_testy.sql -- scenariusze testowe (opcjonalne)
    \item 07\_testy\_rozszerzone.sql -- rozszerzone testy (opcjonalne)
\end{enumerate}

\end{document}
