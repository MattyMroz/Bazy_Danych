\documentclass[12pt,a4paper]{article}

% Pakiety
\usepackage[utf8]{inputenc}
\usepackage[T1]{fontenc}
\usepackage[polish]{babel}
\usepackage{geometry}
\usepackage{graphicx}
\usepackage{float}
\usepackage{booktabs}
\usepackage{longtable}
\usepackage{array}
\usepackage{xcolor}
\usepackage{listings}
\usepackage{hyperref}
\usepackage{fancyhdr}
\usepackage{titlesec}
\usepackage{enumitem}
\usepackage{caption}

% Ustawienia strony
\geometry{
    left=2.5cm,
    right=2.5cm,
    top=2.5cm,
    bottom=2.5cm
}

% Ustawienia nagłówków i stopek
\pagestyle{fancy}
\fancyhf{}
\fancyhead[L]{Projektowanie i Administracja Baz Danych}
\fancyhead[R]{Mateusz Mróz (251190)}
\fancyfoot[C]{\thepage}

% Styl listingów SQL
\lstdefinestyle{sql}{
    language=SQL,
    basicstyle=\ttfamily\small,
    keywordstyle=\color{blue}\bfseries,
    stringstyle=\color{red},
    commentstyle=\color{gray}\itshape,
    showstringspaces=false,
    breaklines=true,
    frame=single,
    backgroundcolor=\color{gray!10},
    numbers=left,
    numberstyle=\tiny\color{gray},
    numbersep=5pt,
    tabsize=4
}

\lstset{style=sql}

% Kolory dla diagramu
\definecolor{tableheader}{RGB}{70,130,180}
\definecolor{tablebody}{RGB}{240,248,255}

\begin{document}

% ============================================================================
% STRONA TYTUŁOWA
% ============================================================================
\begin{titlepage}
    \centering
    \vspace*{2cm}
    
    {\Large\textbf{POLITECHNIKA ŁÓDZKA}}\\[0.3cm]
    {\large Wydział Elektrotechniki, Elektroniki, Informatyki i Automatyki}\\[0.3cm]
    {\large Instytut Informatyki Stosowanej}\\[2cm]
    
    {\LARGE\textbf{Projektowanie i Administracja Baz Danych}}\\[0.5cm]
    {\Large Semestr 5}\\[2cm]
    
    {\Huge\textbf{Projekt Bazy Danych}}\\[0.3cm]
    {\Large\textbf{Company (CrunchBase)}}\\[3cm]
    
    \begin{flushleft}
        \hspace{8cm}
        \begin{tabular}{ll}
            \textbf{Autor:} & Mateusz Mróz \\
            \textbf{Nr indeksu:} & 251190 \\
            \textbf{Kierunek:} & Informatyka \\
            \textbf{Tryb studiów:} & Stacjonarne \\
        \end{tabular}
    \end{flushleft}
    
    \vfill
    
    {\large Łódź, styczeń 2026}
\end{titlepage}

% Spis treści
\tableofcontents
\newpage

% ============================================================================
% ROZDZIAŁ 1: ZAŁOŻENIA WSTĘPNE
% ============================================================================
\section{Założenia wstępne}

\subsection{Cel projektu}
Celem projektu jest zaprojektowanie i implementacja relacyjnej bazy danych na podstawie danych pochodzących z serwisu CrunchBase. Baza danych ma przechowywać informacje o firmach technologicznych, ich finansowaniu, pracownikach, produktach oraz przejęciach.

\subsection{Źródło danych}
Dane źródłowe pochodzą z pliku JSON zawierającego 6 dokumentów opisujących firmy:
\begin{itemize}
    \item Wetpaint
    \item Facebook
    \item Twitter
    \item Omnidrive
    \item StumbleUpon
    \item Scribd
\end{itemize}

\subsection{Wymagania funkcjonalne}
\begin{enumerate}
    \item Przechowywanie danych o firmach (nazwa, opis, kategoria, data założenia, liczba pracowników)
    \item Rejestracja rund finansowania i inwestycji
    \item Śledzenie relacji między osobami a firmami
    \item Ewidencja produktów i biur firm
    \item Historia przejęć i konkurencji
    \item Przechowywanie kamieni milowych firm
\end{enumerate}

\subsection{Wymagania niefunkcjonalne}
\begin{enumerate}
    \item Normalizacja do trzeciej postaci normalnej (3NF)
    \item Integralność referencyjna poprzez klucze obce
    \item Ograniczenia CHECK dla walidacji danych
    \item Indeksy dla optymalizacji zapytań
    \item System ról i uprawnień (contained users)
\end{enumerate}

\subsection{Technologie}
\begin{itemize}
    \item \textbf{SZBD:} Microsoft SQL Server 2019+
    \item \textbf{Język:} T-SQL
    \item \textbf{Tryb bazy:} CONTAINMENT = PARTIAL (dla contained users)
\end{itemize}

\newpage

% ============================================================================
% ROZDZIAŁ 2: DIAGRAM ERD
% ============================================================================
\section{Diagram ERD}

\subsection{Opis encji}
Baza danych składa się z 18 tabel reprezentujących następujące encje:

\begin{table}[H]
\centering
\caption{Lista tabel w bazie danych}
\begin{tabular}{|l|l|p{7cm}|}
\hline
\textbf{Tabela} & \textbf{Opis} & \textbf{Klucz główny} \\
\hline
Company & Firmy & company\_id (INT IDENTITY) \\
Person & Osoby (założyciele, inwestorzy) & person\_id (INT IDENTITY) \\
FinancialOrg & Organizacje finansowe & financial\_org\_id (INT IDENTITY) \\
Product & Produkty firm & product\_id (INT IDENTITY) \\
Office & Biura firm & office\_id (INT IDENTITY) \\
FundingRound & Rundy finansowania & funding\_round\_id (INT IDENTITY) \\
Investment & Inwestycje w rundach & investment\_id (INT IDENTITY) \\
Acquisition & Przejęcia firm & acquisition\_id (INT IDENTITY) \\
Milestone & Kamienie milowe & milestone\_id (INT IDENTITY) \\
Competitor & Konkurenci & competitor\_id (INT IDENTITY) \\
CompanyRelationship & Relacje osoba-firma & relationship\_id (INT IDENTITY) \\
ExternalLink & Linki zewnętrzne & external\_link\_id (INT IDENTITY) \\
Screenshot & Zrzuty ekranu & screenshot\_id (INT IDENTITY) \\
ScreenshotSize & Rozmiary zrzutów & screenshot\_size\_id (INT IDENTITY) \\
VideoEmbed & Osadzone filmy & video\_embed\_id (INT IDENTITY) \\
Provider & Dostawcy usług & provider\_id (INT IDENTITY) \\
CompanyImage & Obrazy firm & company\_image\_id (INT IDENTITY) \\
CompanyIPO & Oferty publiczne (IPO) & ipo\_id (INT IDENTITY) \\
\hline
\end{tabular}
\end{table}

\subsection{Diagram relacji}
Poniżej przedstawiono uproszczony diagram relacji między głównymi encjami:

\begin{figure}[H]
\centering
\begin{verbatim}
                    +----------------+
                    |    Company     |
                    +----------------+
                    | company_id PK  |
                    | name           |
                    | permalink UK   |
                    | category_code  |
                    | ...            |
                    +-------+--------+
                            |
        +-------------------+-------------------+
        |           |           |               |
        v           v           v               v
+-------+------+ +--+-------+ +-+--------+ +----+------+
|   Product    | |  Office  | |FundingRnd| |Competitor |
+--------------+ +----------+ +----------+ +-----------+
| product_id PK| |office_id | |fund_rd_id| |comp_id PK |
| company_id FK| |comp_id FK| |comp_id FK| |comp_id FK |
| name         | | city     | |amount    | |competitor |
+--------------+ +----------+ +----+-----+ +-----------+
                                   |
                                   v
                            +------+-------+
                            |  Investment  |
                            +--------------+
                            |investment_id |
                            |fund_round FK |
                            |person_id FK  |
                            |fin_org_id FK |
                            |inv_comp FK   |
                            +--------------+
                                   |
            +----------------------+----------------------+
            |                      |                      |
            v                      v                      v
     +------+------+      +--------+------+      +--------+-----+
     |   Person    |      | FinancialOrg  |      |   Company    |
     +-------------+      +---------------+      | (inwestor)   |
     |person_id PK |      |fin_org_id PK  |      +--------------+
     |first_name   |      |name           |
     |last_name    |      |permalink UK   |
     |permalink UK |      +---------------+
     +------+------+
            |
            v
     +------+------------+
     |CompanyRelationship|
     +-------------------+
     |relationship_id PK |
     |company_id FK      |
     |person_id FK       |
     |title              |
     |is_past            |
     +-------------------+
\end{verbatim}
\caption{Uproszczony diagram relacji (format ASCII)}
\end{figure}

\subsection{Relacje między tabelami}
\begin{itemize}
    \item \textbf{Company - Product}: 1:N (firma może mieć wiele produktów)
    \item \textbf{Company - Office}: 1:N (firma może mieć wiele biur)
    \item \textbf{Company - FundingRound}: 1:N (firma może mieć wiele rund finansowania)
    \item \textbf{FundingRound - Investment}: 1:N (runda może mieć wielu inwestorów)
    \item \textbf{Person - Investment}: 1:N (osoba może inwestować w wiele rund)
    \item \textbf{FinancialOrg - Investment}: 1:N (org. finansowa może inwestować w wiele rund)
    \item \textbf{Company - CompanyRelationship - Person}: M:N (osoby pracują w firmach)
    \item \textbf{Company - Competitor - Company}: M:N (firmy konkurują ze sobą)
    \item \textbf{Company - Acquisition - Company}: 1:N (firmy przejmują inne firmy)
\end{itemize}

\newpage

% ============================================================================
% ROZDZIAŁ 3: OPIS OBIEKTÓW BAZY DANYCH
% ============================================================================
\section{Opis obiektów bazy danych}

\subsection{Tabele}

\subsubsection{Tabela Company}
Główna tabela przechowująca informacje o firmach.

\begin{lstlisting}[caption={Struktura tabeli Company}]
CREATE TABLE crunchbase.Company (
    company_id INT IDENTITY(1,1) PRIMARY KEY,
    mongo_id VARCHAR(50) NULL,
    name NVARCHAR(255) NOT NULL,
    permalink NVARCHAR(255) NOT NULL UNIQUE,
    crunchbase_url NVARCHAR(500) NULL,
    homepage_url NVARCHAR(500) NULL,
    category_code VARCHAR(100) NULL,
    description NVARCHAR(MAX) NULL,
    number_of_employees INT NULL CHECK (number_of_employees >= 0),
    founded_year INT NULL,
    founded_month INT NULL CHECK (founded_month BETWEEN 1 AND 12),
    founded_day INT NULL CHECK (founded_day BETWEEN 1 AND 31),
    total_money_raised VARCHAR(100) NULL,
    created_at DATETIME DEFAULT GETDATE(),
    updated_at DATETIME DEFAULT GETDATE()
);
\end{lstlisting}

\textbf{Ograniczenia:}
\begin{itemize}
    \item PRIMARY KEY na company\_id
    \item UNIQUE na permalink
    \item CHECK na number\_of\_employees >= 0
    \item CHECK na founded\_month (1-12)
    \item CHECK na founded\_day (1-31)
    \item DEFAULT GETDATE() dla created\_at i updated\_at
\end{itemize}

\subsubsection{Tabela FundingRound}
Przechowuje informacje o rundach finansowania.

\begin{lstlisting}[caption={Struktura tabeli FundingRound}]
CREATE TABLE crunchbase.FundingRound (
    funding_round_id INT IDENTITY(1,1) PRIMARY KEY,
    company_id INT NOT NULL FOREIGN KEY 
        REFERENCES crunchbase.Company(company_id) ON DELETE CASCADE,
    original_id INT NULL,
    round_code VARCHAR(50) NULL,
    raised_amount DECIMAL(18,2) NULL,
    raised_currency_code VARCHAR(10) DEFAULT 'USD',
    funded_year INT NULL,
    funded_month INT NULL,
    funded_day INT NULL
);
\end{lstlisting}

\subsubsection{Tabela Investment}
Tabela asocjacyjna dla inwestycji w rundach finansowania.

\begin{lstlisting}[caption={Struktura tabeli Investment}]
CREATE TABLE crunchbase.Investment (
    investment_id INT IDENTITY(1,1) PRIMARY KEY,
    funding_round_id INT NOT NULL FOREIGN KEY 
        REFERENCES crunchbase.FundingRound(funding_round_id) ON DELETE CASCADE,
    person_id INT NULL FOREIGN KEY 
        REFERENCES crunchbase.Person(person_id),
    financial_org_id INT NULL FOREIGN KEY 
        REFERENCES crunchbase.FinancialOrg(financial_org_id),
    investing_company_id INT NULL FOREIGN KEY 
        REFERENCES crunchbase.Company(company_id)
);
\end{lstlisting}

\subsection{Indeksy}

\begin{lstlisting}[caption={Definicje indeksów}]
CREATE NONCLUSTERED INDEX IX_Company_Name 
    ON crunchbase.Company(name);
CREATE NONCLUSTERED INDEX IX_Company_CategoryCode 
    ON crunchbase.Company(category_code);
CREATE NONCLUSTERED INDEX IX_Company_FoundedYear 
    ON crunchbase.Company(founded_year);
CREATE NONCLUSTERED INDEX IX_Person_LastName 
    ON crunchbase.Person(last_name);
CREATE NONCLUSTERED INDEX IX_FundingRound_RoundCode 
    ON crunchbase.FundingRound(round_code);
CREATE NONCLUSTERED INDEX IX_FundingRound_RaisedAmount 
    ON crunchbase.FundingRound(raised_amount);
CREATE NONCLUSTERED INDEX IX_Office_City 
    ON crunchbase.Office(city);
\end{lstlisting}

\subsection{Procedury składowane}

\subsubsection{AddCompany}
Procedura dodająca nową firmę do bazy danych.

\begin{lstlisting}[caption={Procedura AddCompany}]
CREATE PROCEDURE crunchbase.AddCompany
    @name NVARCHAR(255),
    @permalink NVARCHAR(255),
    @category_code VARCHAR(100) = NULL,
    @description NVARCHAR(MAX) = NULL,
    @founded_year INT = NULL,
    @new_company_id INT OUTPUT
AS
BEGIN
    -- Walidacja uniqueness permalink
    IF EXISTS (SELECT 1 FROM crunchbase.Company 
               WHERE permalink = @permalink)
    BEGIN
        RAISERROR('Firma o podanym permalink juz istnieje!', 16, 1);
        RETURN;
    END
    
    INSERT INTO crunchbase.Company (name, permalink, category_code, 
                                    description, founded_year)
    VALUES (@name, @permalink, @category_code, 
            @description, @founded_year);
    
    SET @new_company_id = SCOPE_IDENTITY();
END
\end{lstlisting}

\subsubsection{SearchCompanies}
Procedura wyszukująca firmy według różnych kryteriów.

\begin{lstlisting}[caption={Procedura SearchCompanies}]
CREATE PROCEDURE crunchbase.SearchCompanies
    @search_name NVARCHAR(255) = NULL,
    @category_code VARCHAR(100) = NULL,
    @min_employees INT = NULL,
    @founded_year_from INT = NULL,
    @has_funding BIT = NULL
AS
BEGIN
    SELECT c.company_id, c.name, c.permalink, c.category_code,
           c.number_of_employees, c.founded_year,
           (SELECT COUNT(*) FROM crunchbase.FundingRound fr 
            WHERE fr.company_id = c.company_id) AS funding_rounds
    FROM crunchbase.Company c
    WHERE (@search_name IS NULL OR c.name LIKE '%' + @search_name + '%')
      AND (@category_code IS NULL OR c.category_code = @category_code)
      AND (@min_employees IS NULL OR c.number_of_employees >= @min_employees)
      AND (@founded_year_from IS NULL OR c.founded_year >= @founded_year_from)
    ORDER BY c.name;
END
\end{lstlisting}

\subsubsection{GetCompanyFundingReport}
Procedura generująca raport finansowania firmy.

\begin{lstlisting}[caption={Procedura GetCompanyFundingReport}]
CREATE PROCEDURE crunchbase.GetCompanyFundingReport
    @company_id INT
AS
BEGIN
    -- Informacje o firmie
    SELECT c.name, c.category_code, c.founded_year, 
           c.number_of_employees, c.total_money_raised
    FROM crunchbase.Company c WHERE c.company_id = @company_id;
    
    -- Rundy finansowania
    SELECT fr.round_code, fr.raised_amount, fr.raised_currency_code,
           fr.funded_year, fr.funded_month
    FROM crunchbase.FundingRound fr
    WHERE fr.company_id = @company_id
    ORDER BY fr.funded_year DESC;
    
    -- Podsumowanie
    SELECT COUNT(*) AS LiczbaRund, SUM(raised_amount) AS SumaFinansowania,
           AVG(raised_amount) AS SredniaKwota
    FROM crunchbase.FundingRound WHERE company_id = @company_id;
END
\end{lstlisting}

\subsection{Funkcje}

\subsubsection{GetTotalFunding (skalarna)}
Oblicza całkowite finansowanie firmy.

\begin{lstlisting}[caption={Funkcja GetTotalFunding}]
CREATE FUNCTION crunchbase.GetTotalFunding(@company_id INT)
RETURNS DECIMAL(18,2)
AS
BEGIN
    DECLARE @total DECIMAL(18,2);
    SELECT @total = ISNULL(SUM(raised_amount), 0)
    FROM crunchbase.FundingRound
    WHERE company_id = @company_id;
    RETURN @total;
END
\end{lstlisting}

\subsubsection{GetCompanyAge (skalarna)}
Oblicza wiek firmy w latach.

\begin{lstlisting}[caption={Funkcja GetCompanyAge}]
CREATE FUNCTION crunchbase.GetCompanyAge(@company_id INT)
RETURNS INT
AS
BEGIN
    DECLARE @age INT, @founded_year INT;
    SELECT @founded_year = founded_year
    FROM crunchbase.Company WHERE company_id = @company_id;
    
    IF @founded_year IS NULL SET @age = NULL;
    ELSE SET @age = YEAR(GETDATE()) - @founded_year;
    RETURN @age;
END
\end{lstlisting}

\subsubsection{GetCompanyPeople (tabelaryczna)}
Zwraca listę osób powiązanych z firmą.

\begin{lstlisting}[caption={Funkcja GetCompanyPeople}]
CREATE FUNCTION crunchbase.GetCompanyPeople(@company_id INT)
RETURNS TABLE
AS
RETURN (
    SELECT p.person_id, p.first_name, p.last_name, p.permalink,
           cr.title AS Stanowisko,
           CASE WHEN cr.is_past = 1 THEN 'Tak' ELSE 'Nie' END AS Byly
    FROM crunchbase.CompanyRelationship cr
    INNER JOIN crunchbase.Person p ON p.person_id = cr.person_id
    WHERE cr.company_id = @company_id
);
\end{lstlisting}

\subsection{Widoki}

\subsubsection{vw\_CompanyOverview}
Widok pokazujący przegląd firm z podstawowymi statystykami.

\begin{lstlisting}[caption={Widok vw\_CompanyOverview}]
CREATE VIEW crunchbase.vw_CompanyOverview AS
SELECT 
    c.company_id, c.name AS Nazwa, c.permalink, c.category_code AS Kategoria,
    c.number_of_employees AS LiczbaPracownikow,
    crunchbase.GetCompanyAge(c.company_id) AS WiekFirmy,
    crunchbase.GetTotalFunding(c.company_id) AS CalkowiteFinansowanie,
    (SELECT COUNT(*) FROM crunchbase.FundingRound fr 
     WHERE fr.company_id = c.company_id) AS LiczbaRund,
    (SELECT COUNT(*) FROM crunchbase.Product p 
     WHERE p.company_id = c.company_id) AS LiczbaProdukow
FROM crunchbase.Company c;
\end{lstlisting}

\subsubsection{vw\_FundingByCategory}
Widok pokazujący statystyki finansowania według kategorii.

\begin{lstlisting}[caption={Widok vw\_FundingByCategory}]
CREATE VIEW crunchbase.vw_FundingByCategory AS
SELECT 
    c.category_code AS Kategoria,
    COUNT(DISTINCT c.company_id) AS LiczbaFirm,
    COUNT(fr.funding_round_id) AS LiczbaRund,
    SUM(fr.raised_amount) AS SumaFinansowania,
    AVG(fr.raised_amount) AS SredniaKwota
FROM crunchbase.Company c
LEFT JOIN crunchbase.FundingRound fr ON fr.company_id = c.company_id
WHERE c.category_code IS NOT NULL
GROUP BY c.category_code;
\end{lstlisting}

\subsubsection{vw\_TopInvestors}
Widok pokazujący najbardziej aktywnych inwestorów.

\begin{lstlisting}[caption={Widok vw\_TopInvestors}]
CREATE VIEW crunchbase.vw_TopInvestors AS
SELECT InvestorType, InvestorName, InvestorPermalink,
       InvestmentCount, TotalCompaniesInvested
FROM (
    -- Osoby
    SELECT 'Osoba' AS InvestorType, 
           CONCAT(p.first_name, ' ', p.last_name) AS InvestorName,
           p.permalink AS InvestorPermalink, COUNT(*) AS InvestmentCount,
           COUNT(DISTINCT fr.company_id) AS TotalCompaniesInvested
    FROM crunchbase.Investment inv
    INNER JOIN crunchbase.Person p ON p.person_id = inv.person_id
    INNER JOIN crunchbase.FundingRound fr 
        ON fr.funding_round_id = inv.funding_round_id
    GROUP BY p.person_id, p.first_name, p.last_name, p.permalink
    UNION ALL
    -- Organizacje finansowe
    SELECT 'Organizacja' AS InvestorType, fo.name, fo.permalink,
           COUNT(*), COUNT(DISTINCT fr.company_id)
    FROM crunchbase.Investment inv
    INNER JOIN crunchbase.FinancialOrg fo 
        ON fo.financial_org_id = inv.financial_org_id
    INNER JOIN crunchbase.FundingRound fr 
        ON fr.funding_round_id = inv.funding_round_id
    GROUP BY fo.financial_org_id, fo.name, fo.permalink
) AS AllInvestors;
\end{lstlisting}

\subsection{Triggery}

\subsubsection{trg\_Company\_UpdateTimestamp}
Automatyczna aktualizacja znacznika czasu przy modyfikacji firmy.

\begin{lstlisting}[caption={Trigger trg\_Company\_UpdateTimestamp}]
CREATE TRIGGER crunchbase.trg_Company_UpdateTimestamp
ON crunchbase.Company
AFTER UPDATE
AS
BEGIN
    SET NOCOUNT ON;
    UPDATE crunchbase.Company
    SET updated_at = GETDATE()
    FROM crunchbase.Company c
    INNER JOIN inserted i ON c.company_id = i.company_id
    WHERE c.updated_at = i.updated_at;
END
\end{lstlisting}

\subsubsection{trg\_FundingRound\_Audit}
Trigger audytowy rejestrujący zmiany w rundach finansowania.

\begin{lstlisting}[caption={Trigger trg\_FundingRound\_Audit}]
CREATE TRIGGER crunchbase.trg_FundingRound_Audit
ON crunchbase.FundingRound
AFTER INSERT, UPDATE, DELETE
AS
BEGIN
    SET NOCOUNT ON;
    
    -- INSERT
    IF EXISTS (SELECT 1 FROM inserted) AND NOT EXISTS (SELECT 1 FROM deleted)
        INSERT INTO crunchbase.FundingRound_Audit 
            (funding_round_id, company_id, action_type, 
             new_raised_amount, new_round_code)
        SELECT funding_round_id, company_id, 'INSERT', 
               raised_amount, round_code FROM inserted;
    
    -- UPDATE
    IF EXISTS (SELECT 1 FROM inserted) AND EXISTS (SELECT 1 FROM deleted)
        INSERT INTO crunchbase.FundingRound_Audit
            (funding_round_id, company_id, action_type,
             old_raised_amount, new_raised_amount, old_round_code, new_round_code)
        SELECT i.funding_round_id, i.company_id, 'UPDATE',
               d.raised_amount, i.raised_amount, d.round_code, i.round_code
        FROM inserted i INNER JOIN deleted d 
            ON i.funding_round_id = d.funding_round_id;
    
    -- DELETE
    IF NOT EXISTS (SELECT 1 FROM inserted) AND EXISTS (SELECT 1 FROM deleted)
        INSERT INTO crunchbase.FundingRound_Audit
            (funding_round_id, company_id, action_type, 
             old_raised_amount, old_round_code)
        SELECT funding_round_id, company_id, 'DELETE', 
               raised_amount, round_code FROM deleted;
END
\end{lstlisting}

\subsubsection{trg\_Investment\_Validate}
Trigger walidacyjny sprawdzający poprawność inwestycji.

\begin{lstlisting}[caption={Trigger trg\_Investment\_Validate}]
CREATE TRIGGER crunchbase.trg_Investment_Validate
ON crunchbase.Investment
INSTEAD OF INSERT
AS
BEGIN
    SET NOCOUNT ON;
    
    -- Sprawdzenie czy inwestor jest podany
    IF EXISTS (SELECT 1 FROM inserted 
               WHERE person_id IS NULL 
                 AND financial_org_id IS NULL 
                 AND investing_company_id IS NULL)
    BEGIN
        RAISERROR('Inwestycja musi miec inwestora!', 16, 1);
        RETURN;
    END
    
    -- Wstawienie zatwierdzonych rekordow
    INSERT INTO crunchbase.Investment 
        (funding_round_id, person_id, financial_org_id, investing_company_id)
    SELECT funding_round_id, person_id, financial_org_id, investing_company_id
    FROM inserted;
END
\end{lstlisting}

\newpage

% ============================================================================
% ROZDZIAŁ 4: ROLE, UPRAWNIENIA, UŻYTKOWNICY
% ============================================================================
\section{Role, uprawnienia i użytkownicy}

\subsection{Architektura bezpieczeństwa}
Baza danych wykorzystuje mechanizm \textbf{contained users} (użytkownicy zawierci), co umożliwia:
\begin{itemize}
    \item Logowanie bezpośrednio do bazy danych bez potrzeby konta na serwerze
    \item Łatwiejszą przenośność bazy danych między serwerami
    \item Izolację uprawnień na poziomie bazy danych
\end{itemize}

\subsection{Role}
W bazie danych zdefiniowano 5 ról:

\begin{table}[H]
\centering
\caption{Zdefiniowane role i ich uprawnienia}
\begin{tabular}{|l|p{8cm}|}
\hline
\textbf{Rola} & \textbf{Uprawnienia} \\
\hline
role\_Admin & Pełna kontrola (db\_owner) - wszystkie operacje \\
\hline
role\_Employee & EXECUTE + SELECT na schemacie crunchbase \\
\hline
role\_Guest & SELECT tylko na widokach \\
\hline
role\_DataAnalyst & SELECT + EXECUTE (bez modyfikacji danych) \\
\hline
role\_ReportViewer & SELECT na widokach + wybrane procedury raportowe \\
\hline
\end{tabular}
\end{table}

\subsection{Użytkownicy}
\begin{table}[H]
\centering
\caption{Użytkownicy contained i ich role}
\begin{tabular}{|l|l|l|}
\hline
\textbf{Użytkownik} & \textbf{Rola} & \textbf{Opis} \\
\hline
admin\_user & role\_Admin & Administrator z pełnym dostępem \\
emp\_user & role\_Employee & Pracownik - procedury i odczyt \\
guest\_user & role\_Guest & Gość - tylko widoki \\
analyst\_user & role\_DataAnalyst & Analityk danych \\
report\_user & role\_ReportViewer & Przeglądanie raportów \\
\hline
\end{tabular}
\end{table}

\subsection{Kod tworzący role i użytkowników}

\begin{lstlisting}[caption={Tworzenie ról}]
-- Rola Admin
CREATE ROLE role_Admin;
ALTER ROLE db_owner ADD MEMBER role_Admin;

-- Rola Employee
CREATE ROLE role_Employee;
GRANT EXECUTE ON SCHEMA::crunchbase TO role_Employee;
GRANT SELECT ON SCHEMA::crunchbase TO role_Employee;

-- Rola Guest
CREATE ROLE role_Guest;
GRANT SELECT ON crunchbase.vw_CompanyOverview TO role_Guest;
GRANT SELECT ON crunchbase.vw_FundingByCategory TO role_Guest;
GRANT SELECT ON crunchbase.vw_TopInvestors TO role_Guest;
DENY SELECT ON crunchbase.Company TO role_Guest;
DENY INSERT, UPDATE, DELETE ON SCHEMA::crunchbase TO role_Guest;
\end{lstlisting}

\begin{lstlisting}[caption={Tworzenie użytkowników contained}]
-- Administrator
CREATE USER admin_user WITH PASSWORD = 'Admin123!@#Strong';
ALTER ROLE role_Admin ADD MEMBER admin_user;

-- Pracownik
CREATE USER emp_user WITH PASSWORD = 'Emp456!@#Strong';
ALTER ROLE role_Employee ADD MEMBER emp_user;

-- Gość
CREATE USER guest_user WITH PASSWORD = 'Guest789!@#Strong';
ALTER ROLE role_Guest ADD MEMBER guest_user;
\end{lstlisting}

\subsection{Instrukcja logowania}
Aby zalogować się jako użytkownik contained w SQL Server Management Studio:
\begin{enumerate}
    \item Server: localhost (lub nazwa serwera)
    \item Authentication: SQL Server Authentication
    \item Login: nazwa użytkownika (np. admin\_user)
    \item Password: hasło użytkownika
    \item Options $\rightarrow$ Connection Properties $\rightarrow$ Connect to database: CompanyDB
\end{enumerate}

\newpage

% ============================================================================
% ROZDZIAŁ 5: UWAGI KOŃCOWE
% ============================================================================
\section{Uwagi końcowe}

\subsection{Normalizacja}
Baza danych została znormalizowana do trzeciej postaci normalnej (3NF):
\begin{itemize}
    \item \textbf{1NF}: Wszystkie atrybuty są atomowe, każda tabela ma klucz główny
    \item \textbf{2NF}: Usunięto częściowe zależności funkcyjne - wszystkie atrybuty nie będące kluczem zależą od całego klucza głównego
    \item \textbf{3NF}: Usunięto zależności przechodnie - atrybuty nie zależą od innych atrybutów nie będących kluczem
\end{itemize}

\textbf{Wyjątek od 3NF:} Pole \texttt{tag\_list} w tabeli Company zostało celowo pozostawione jako VARCHAR z listą tagów oddzielonych przecinkami. Decyzja ta wynika z:
\begin{itemize}
    \item Rzadkiego użycia do wyszukiwania
    \item Uproszczenia struktury bazy
    \item Zgodności z formatem danych źródłowych
\end{itemize}

\subsection{Decyzje projektowe}

\subsubsection{Daty jako osobne kolumny}
Daty (np. founded\_year, founded\_month, founded\_day) zostały rozdzielone na osobne kolumny INT zamiast typu DATE, ponieważ:
\begin{itemize}
    \item Dane źródłowe często zawierają niepełne daty (tylko rok lub rok i miesiąc)
    \item Typ DATE wymaga kompletnej daty
    \item Rozdzielenie umożliwia elastyczne zapytania (np. firmy założone w danym roku)
\end{itemize}

\subsubsection{Nullable foreign keys}
Niektóre klucze obce dopuszczają wartości NULL, ponieważ:
\begin{itemize}
    \item Inwestor może być osobą, organizacją finansową LUB firmą
    \item Konkurent może nie istnieć jako osobna firma w bazie
    \item Firma przejmująca może nie być w naszym zbiorze danych
\end{itemize}

\subsubsection{ON DELETE CASCADE}
Kaskadowe usuwanie zostało włączone dla relacji:
\begin{itemize}
    \item FundingRound $\rightarrow$ Investment
    \item Company $\rightarrow$ Product, Office, FundingRound, Milestone, etc.
\end{itemize}

\subsection{Możliwe rozszerzenia}
\begin{enumerate}
    \item Dodanie pełnotekstowego wyszukiwania (Full-Text Search) dla opisu i overview
    \item Implementacja historii zmian dla wszystkich tabel (temporal tables)
    \item Partycjonowanie dużych tabel według roku
    \item Dodanie warstwy API (REST API) do komunikacji z bazą
    \item Integracja z narzędziami BI (Power BI, Tableau)
\end{enumerate}

\subsection{Podsumowanie}
Projekt obejmuje:
\begin{itemize}
    \item 18 tabel relacyjnych
    \item 7 indeksów niesklastrowanych
    \item 5 procedur składowanych
    \item 6 funkcji (3 skalarne, 3 tabelaryczne)
    \item 6 widoków
    \item 4 triggery
    \item 5 ról i 5 użytkowników contained
    \item Tabela audytu dla śledzenia zmian
\end{itemize}

Baza danych jest gotowa do użycia produkcyjnego i może być rozszerzana według potrzeb biznesowych.

\end{document}
