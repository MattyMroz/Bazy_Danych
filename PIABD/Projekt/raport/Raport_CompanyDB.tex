% ============================================================================
% RAPORT PROJEKTU BAZY DANYCH - CompanyDB (CrunchBase)
% Autor: Mateusz Mróz (251190)
% Przedmiot: Projektowanie i Administracja Baz Danych
% ============================================================================

\documentclass[12pt,a4paper]{article}

% --- PAKIETY ---
\usepackage[utf8]{inputenc}
\usepackage[T1]{fontenc}
\usepackage[polish]{babel}
\usepackage{geometry}
\usepackage{graphicx}
\usepackage{listings}
\usepackage{xcolor}
\usepackage{hyperref}
\usepackage{booktabs}
\usepackage{array}
\usepackage{longtable}
\usepackage{float}
\usepackage{caption}
\usepackage{fancyhdr}
\usepackage{titlesec}
\usepackage{parskip}
\usepackage{enumitem}
\usepackage{amssymb}
\usepackage{tabularx}

% --- GEOMETRIA ---
\geometry{
    a4paper,
    left=2.5cm,
    right=2.5cm,
    top=2.5cm,
    bottom=2.5cm,
    headheight=14pt
}

% --- NAGŁÓWKI I STOPKI ---
\pagestyle{fancy}
\fancyhf{}
\fancyhead[L]{\small Projekt Bazy Danych --- CompanyDB}
\fancyhead[R]{\small Mateusz Mróz (251190)}
\fancyfoot[C]{\thepage}
\renewcommand{\headrulewidth}{0.4pt}
\renewcommand{\footrulewidth}{0.4pt}

% --- STYL KODU SQL ---
\definecolor{sqlblue}{RGB}{0,0,180}
\definecolor{sqlgreen}{RGB}{0,128,0}
\definecolor{sqlgray}{RGB}{128,128,128}
\definecolor{sqlbg}{RGB}{248,248,248}

\lstdefinestyle{sqlstyle}{
    language=SQL,
    basicstyle=\ttfamily\small,
    keywordstyle=\color{sqlblue}\bfseries,
    commentstyle=\color{sqlgreen}\itshape,
    stringstyle=\color{red},
    showstringspaces=false,
    breaklines=true,
    frame=single,
    backgroundcolor=\color{sqlbg},
    numbers=left,
    numberstyle=\tiny\color{sqlgray},
    numbersep=8pt,
    tabsize=4,
    captionpos=b,
    morekeywords={NVARCHAR,VARCHAR,DECIMAL,BIT,IDENTITY,SCHEMA,GO,USE,EXEC,ISNULL,TRY_CAST,OPENJSON,GETDATE,DATETIME2}
}
\lstset{style=sqlstyle}

% --- HYPERREF ---
\hypersetup{
    colorlinks=true,
    linkcolor=black,
    filecolor=magenta,
    urlcolor=blue,
    pdftitle={Projekt Bazy Danych --- CompanyDB},
    pdfauthor={Mateusz Mróz},
}

% --- TYTUŁY SEKCJI ---
\titleformat{\section}{\Large\bfseries}{\thesection.}{0.5em}{}
\titleformat{\subsection}{\large\bfseries}{\thesubsection.}{0.5em}{}
\titleformat{\subsubsection}{\normalsize\bfseries}{\thesubsubsection.}{0.5em}{}

% ============================================================================
% POCZĄTEK DOKUMENTU
% ============================================================================
\begin{document}

% ============================================================================
% STRONA TYTUŁOWA
% ============================================================================
\begin{titlepage}
    \centering
    \vspace*{1.5cm}
    
    {\Large\textsc{POLITECHNIKA ŁÓDZKA}\\[0.4cm]
    \large Wydział Elektrotechniki, Elektroniki, Informatyki i Automatyki}
    
    \vspace{2.5cm}
    
    {\large\textbf{Projektowanie i Administracja Baz Danych}\\[0.2cm]
    Semestr 5}
    
    \vspace{2.5cm}
    
    \rule{0.8\textwidth}{0.5pt}
    \vspace{0.5cm}
    
    {\Huge\bfseries Projekt Bazy Danych\\[0.4cm]}
    {\LARGE Company (CrunchBase)}
    
    \vspace{0.5cm}
    \rule{0.8\textwidth}{0.5pt}
    
    \vspace{3.5cm}
    
    \begin{flushright}
        \large
        \textbf{Autor:} Mateusz Mróz\\
        \textbf{Nr indeksu:} 251190
    \end{flushright}
    
    \vfill
    
    {\large Łódź, styczeń 2026}
    
\end{titlepage}

% ============================================================================
% SPIS TREŚCI
% ============================================================================
\tableofcontents
\newpage

% ============================================================================
% ROZDZIAŁ 1: PODSTAWOWE ZAŁOŻENIA PROJEKTU
% ============================================================================
\section{Podstawowe założenia projektu}

\subsection{Cel tworzenia bazy danych}

Celem projektu jest zaprojektowanie i implementacja relacyjnej bazy danych przechowującej informacje o firmach technologicznych, pochodzące z serwisu CrunchBase. Baza danych ma umożliwiać:

\begin{itemize}[noitemsep]
    \item przechowywanie szczegółowych informacji o firmach (nazwa, opis, kategoria, data założenia),
    \item śledzenie rund finansowania i inwestycji,
    \item zarządzanie informacjami o osobach związanych z firmami (założyciele, pracownicy, inwestorzy),
    \item rejestrowanie przejęć, konkurentów i kamieni milowych w historii firm,
    \item przechowywanie danych o produktach, biurach i mediach firmowych.
\end{itemize}

\subsection{Główne założenia}

\begin{enumerate}[noitemsep]
    \item \textbf{Normalizacja} --- baza danych spełnia trzy pierwsze postacie normalne (1NF, 2NF, 3NF).
    \item \textbf{Integralność danych} --- wykorzystanie kluczy głównych, obcych oraz ograniczeń CHECK, UNIQUE i DEFAULT.
    \item \textbf{Bezpieczeństwo} --- system ról i użytkowników z różnymi poziomami uprawnień (contained users).
    \item \textbf{Skalowalność} --- struktura pozwalająca na łatwe dodawanie nowych firm i danych.
    \item \textbf{Zgodność z danymi źródłowymi} --- nazwy kolumn zachowują oryginalne nazwy pól z dokumentów JSON.
\end{enumerate}

\subsection{Zakres możliwości systemu}

System umożliwia:

\begin{itemize}[noitemsep]
    \item import danych z plików JSON przy użyciu funkcji OPENJSON,
    \item wstawianie i aktualizację firm przez procedurę składowaną \texttt{UpsertCompany},
    \item obliczanie łącznego finansowania firmy przez funkcję skalarną \texttt{GetTotalFunding},
    \item przeglądanie podsumowania firm przez widok \texttt{vw\_CompanyOverview},
    \item zarządzanie dostępem przez role i contained users.
\end{itemize}

\subsection{Ograniczenia przyjęte podczas projektowania}

\begin{itemize}[noitemsep]
    \item Dane źródłowe ograniczone do 6 dokumentów JSON (6 firm: Wetpaint, Facebook, Twitter, Omnidrive, StumbleUpon, Scribd).
    \item Brak walidacji zewnętrznych URL-i (przechowywane jako tekst).
    \item Obrazy przechowywane jako ścieżki do plików, nie jako dane binarne (BLOB).
    \item Waluta finansowania domyślnie USD (brak automatycznego przeliczania walut).
    \item Baza typu \texttt{CONTAINMENT = PARTIAL} (umożliwia tworzenie contained users).
\end{itemize}

\newpage
% ============================================================================
% ROZDZIAŁ 2: SCHEMAT BAZY DANYCH
% ============================================================================
\section{Schemat bazy danych}

\subsection{Diagram ERD}

Poniżej przedstawiono diagram relacji bazy danych. Diagram można wygenerować w SQL Server Management Studio poprzez: Database Diagrams $\rightarrow$ New Database Diagram.

\begin{figure}[H]
    \centering
    \fbox{\parbox{0.95\textwidth}{
        \centering
        \vspace{3cm}
        \textit{[Miejsce na diagram ERD wygenerowany z SSMS]}
        \vspace{3cm}
    }}
    \caption{Diagram relacji bazy danych CompanyDB}
    \label{fig:erd}
\end{figure}

\subsection{Opis struktury}

Baza danych składa się z \textbf{18 tabel} zorganizowanych w schemacie \texttt{crunchbase}. Tabele można podzielić na następujące grupy logiczne:

\begin{table}[H]
\centering
\caption{Podział tabel na grupy logiczne}
\begin{tabular}{|l|l|}
\hline
\textbf{Grupa} & \textbf{Tabele} \\
\hline
Główna & Company \\
\hline
Podmioty & Person, FinancialOrg \\
\hline
Finansowanie & FundingRound, Investment, CompanyIPO \\
\hline
Struktura firmy & Product, Office, CompanyRelationship \\
\hline
Relacje biznesowe & Competitor, Acquisition, Provider, Milestone \\
\hline
Media & ExternalLink, VideoEmbed, CompanyImage, Screenshot, ScreenshotSize \\
\hline
\end{tabular}
\end{table}

\subsection{Główne relacje między tabelami}

\begin{itemize}[noitemsep]
    \item \textbf{Company 1:N Product} --- firma może mieć wiele produktów
    \item \textbf{Company 1:N Office} --- firma może mieć wiele biur/siedzib
    \item \textbf{Company 1:N FundingRound} --- firma może mieć wiele rund finansowania
    \item \textbf{FundingRound 1:N Investment} --- runda może mieć wiele inwestycji
    \item \textbf{Person M:N Company} --- osoba może być związana z wieloma firmami (przez CompanyRelationship)
    \item \textbf{Company 1:N Acquisition} --- firma może przejmować wiele innych firm
    \item \textbf{Screenshot 1:N ScreenshotSize} --- zrzut ekranu może mieć wiele rozmiarów
\end{itemize}

\newpage
% ============================================================================
% ROZDZIAŁ 3: OBIEKTY BAZY DANYCH
% ============================================================================
\section{Obiekty bazy danych i ich opis}

% --- 3.A TABELE ---
\subsection{Tabele}

\subsubsection{Company --- Tabela główna firm}

\textbf{Opis funkcjonalny:} Przechowuje podstawowe informacje o firmach technologicznych --- nazwę, opis, kategorię działalności, daty założenia i zamknięcia, dane kontaktowe oraz informacje o pozyskanym finansowaniu.

\begin{table}[H]
\centering
\caption{Struktura tabeli Company (wybrane kolumny)}
\small
\begin{tabular}{|l|l|l|}
\hline
\textbf{Kolumna} & \textbf{Typ danych} & \textbf{Ograniczenia} \\
\hline
company\_id & INT IDENTITY(1,1) & PRIMARY KEY \\
mongo\_id & NVARCHAR(50) & NOT NULL, UNIQUE \\
name & NVARCHAR(255) & NOT NULL \\
permalink & NVARCHAR(255) & NOT NULL, UNIQUE \\
category\_code & NVARCHAR(100) & NULL \\
number\_of\_employees & INT & CHECK ($\geq$ 0) \\
founded\_year & INT & NULL \\
founded\_month & INT & CHECK (1-12) \\
founded\_day & INT & CHECK (1-31) \\
created\_at & DATETIME2 & DEFAULT GETDATE() \\
updated\_at & DATETIME2 & DEFAULT GETDATE() \\
\hline
\end{tabular}
\end{table}

\textbf{Klucz główny:} \texttt{company\_id} (auto-inkrementacja)

\textbf{Ograniczenia UNIQUE:} \texttt{mongo\_id}, \texttt{permalink}

\textbf{Ograniczenia CHECK:}
\begin{itemize}[noitemsep]
    \item \texttt{number\_of\_employees >= 0}
    \item \texttt{founded\_month BETWEEN 1 AND 12}
    \item \texttt{founded\_day BETWEEN 1 AND 31}
\end{itemize}

\textbf{Ograniczenia DEFAULT:}
\begin{itemize}[noitemsep]
    \item \texttt{created\_at = GETDATE()}
    \item \texttt{updated\_at = GETDATE()}
\end{itemize}

\subsubsection{Person --- Osoby}

\textbf{Opis funkcjonalny:} Przechowuje informacje o osobach związanych z firmami --- założycielach, pracownikach, inwestorach.

\begin{table}[H]
\centering
\caption{Struktura tabeli Person}
\begin{tabular}{|l|l|l|}
\hline
\textbf{Kolumna} & \textbf{Typ danych} & \textbf{Ograniczenia} \\
\hline
person\_id & INT IDENTITY(1,1) & PRIMARY KEY \\
first\_name & NVARCHAR(100) & NOT NULL \\
last\_name & NVARCHAR(100) & NOT NULL \\
permalink & NVARCHAR(255) & NOT NULL, UNIQUE \\
\hline
\end{tabular}
\end{table}

\subsubsection{FinancialOrg --- Organizacje finansowe}

\textbf{Opis funkcjonalny:} Przechowuje informacje o funduszach VC, funduszach inwestycyjnych i innych organizacjach finansowych inwestujących w firmy.

\begin{table}[H]
\centering
\caption{Struktura tabeli FinancialOrg}
\begin{tabular}{|l|l|l|}
\hline
\textbf{Kolumna} & \textbf{Typ danych} & \textbf{Ograniczenia} \\
\hline
financial\_org\_id & INT IDENTITY(1,1) & PRIMARY KEY \\
name & NVARCHAR(255) & NOT NULL \\
permalink & NVARCHAR(255) & NOT NULL, UNIQUE \\
\hline
\end{tabular}
\end{table}

\subsubsection{FundingRound --- Rundy finansowania}

\textbf{Opis funkcjonalny:} Przechowuje informacje o rundach finansowania firm --- typ rundy (seed, Series A, B, C\ldots), kwotę, walutę, datę.

\begin{table}[H]
\centering
\caption{Struktura tabeli FundingRound}
\begin{tabular}{|l|l|l|}
\hline
\textbf{Kolumna} & \textbf{Typ danych} & \textbf{Ograniczenia} \\
\hline
funding\_round\_id & INT IDENTITY(1,1) & PRIMARY KEY \\
company\_id & INT & FK $\rightarrow$ Company, ON DELETE CASCADE \\
round\_code & NVARCHAR(50) & NULL \\
raised\_amount & DECIMAL(18,2) & CHECK ($\geq$ 0) \\
raised\_currency\_code & NVARCHAR(10) & DEFAULT 'USD' \\
funded\_year & INT & NULL \\
funded\_month & INT & CHECK (1-12) \\
funded\_day & INT & CHECK (1-31) \\
\hline
\end{tabular}
\end{table}

\subsubsection{Investment --- Inwestycje}

\textbf{Opis funkcjonalny:} Tabela łącząca inwestorów z rundami finansowania. Inwestorem może być osoba, organizacja finansowa lub inna firma.

\begin{table}[H]
\centering
\caption{Struktura tabeli Investment}
\begin{tabular}{|l|l|l|}
\hline
\textbf{Kolumna} & \textbf{Typ danych} & \textbf{Ograniczenia} \\
\hline
investment\_id & INT IDENTITY(1,1) & PRIMARY KEY \\
funding\_round\_id & INT & FK $\rightarrow$ FundingRound, ON DELETE CASCADE \\
person\_id & INT & FK $\rightarrow$ Person (opcjonalny) \\
financial\_org\_id & INT & FK $\rightarrow$ FinancialOrg (opcjonalny) \\
investing\_company\_id & INT & FK $\rightarrow$ Company (opcjonalny) \\
\hline
\end{tabular}
\end{table}

\textbf{Uwaga:} Inwestorem może być osoba, organizacja finansowa lub firma --- stąd trzy opcjonalne klucze obce.

\subsubsection{Pozostałe tabele}

\textbf{Product} --- produkty oferowane przez firmy (FK do Company z ON DELETE CASCADE).

\textbf{Office} --- lokalizacje biur i siedzib firm (FK do Company z ON DELETE CASCADE).

\textbf{CompanyRelationship} --- tabela asocjacyjna łącząca osoby z firmami, zawiera stanowisko i flagę \texttt{is\_past}.

\textbf{Competitor} --- informacje o konkurentach firm.

\textbf{Acquisition} --- informacje o przejęciach firm (firma przejmująca i przejmowana).

\textbf{Milestone} --- kamienie milowe w historii firmy.

\textbf{Provider} --- dostawcy usług dla firm.

\textbf{ExternalLink} --- linki zewnętrzne do artykułów o firmie.

\textbf{VideoEmbed} --- osadzone filmy (kod embed).

\textbf{CompanyImage} --- obrazy i logo firm.

\textbf{Screenshot} --- zrzuty ekranu produktów.

\textbf{ScreenshotSize} --- różne rozmiary zrzutów ekranu.

\textbf{CompanyIPO} --- informacje o IPO firmy.

\subsubsection{Uzasadnienie normalizacji}

Baza danych spełnia wszystkie trzy postacie normalne:

\textbf{1NF (Pierwsza postać normalna):}
\begin{itemize}[noitemsep]
    \item Każda kolumna zawiera wartości atomowe (niepodzielne).
    \item Tablice z JSON (np. products, offices, funding\_rounds) zostały rozdzielone do osobnych tabel.
    \item Każda tabela ma unikalny klucz główny (IDENTITY).
\end{itemize}

\textbf{2NF (Druga postać normalna):}
\begin{itemize}[noitemsep]
    \item Baza jest w 1NF.
    \item Wszystkie atrybuty niekluczowe zależą od całego klucza głównego.
    \item Nie występują częściowe zależności funkcyjne.
\end{itemize}

\textbf{3NF (Trzecia postać normalna):}
\begin{itemize}[noitemsep]
    \item Baza jest w 2NF.
    \item Nie występują przechodnie zależności funkcyjne.
    \item Każdy atrybut niekluczowy zależy bezpośrednio od klucza głównego.
\end{itemize}

\subsubsection{Odstępstwa od pełnej normalizacji}

W projekcie celowo pozostawiono jedno odstępstwo od pełnej 1NF:

\textbf{Pole tag\_list w tabeli Company:}
\begin{itemize}[noitemsep]
    \item Typ danych: \texttt{NVARCHAR(MAX)}
    \item Format: wartości oddzielone przecinkami (np. ``wiki, seattle, elowitz'')
    \item Odstępstwo: pole zawiera wiele wartości zamiast wartości atomowej
\end{itemize}

\textbf{Uzasadnienie:}
Pole \texttt{tag\_list} pozostawiono jako ciąg tekstowy (lista rozdzielana przecinkami) zamiast wydzielać je do osobnej tabeli \texttt{Tags} z relacją wiele-do-wielu. Decyzja ta wynika z:
\begin{itemize}[noitemsep]
    \item uproszczenia struktury bazy danych,
    \item zachowania zgodności z oryginalnym formatem prezentacji danych,
    \item traktowania tagów jako prostych słów kluczowych do wyszukiwania pełnotekstowego,
    \item braku potrzeby wykonywania złożonych operacji na pojedynczych tagach.
\end{itemize}

W przypadku konieczności pracy z poszczególnymi tagami, można wykorzystać funkcje SQL Server do parsowania ciągów tekstowych (np. \texttt{STRING\_SPLIT}).

% --- 3.B INDEKSY ---
\subsection{Indeksy}

W bazie danych utworzono następujące indeksy nieklastrowe (poza automatycznymi dla PK i UNIQUE):

\begin{table}[H]
\centering
\caption{Indeksy utworzone w bazie danych}
\begin{tabular}{|l|l|l|}
\hline
\textbf{Nazwa indeksu} & \textbf{Tabela} & \textbf{Kolumna} \\
\hline
IX\_Company\_Name & Company & name \\
IX\_Company\_CategoryCode & Company & category\_code \\
IX\_FundingRound\_RoundCode & FundingRound & round\_code \\
\hline
\end{tabular}
\end{table}

\textbf{Uzasadnienie utworzenia indeksów:}

\begin{enumerate}[noitemsep]
    \item \textbf{IX\_Company\_Name} --- przyspiesza wyszukiwanie firm po nazwie, często używane w zapytaniach.
    
    \item \textbf{IX\_Company\_CategoryCode} --- przyspiesza filtrowanie firm według kategorii działalności.
    
    \item \textbf{IX\_FundingRound\_RoundCode} --- przyspiesza grupowanie i filtrowanie rund finansowania według typu (seed, series-a, series-b, etc.).
\end{enumerate}

% --- 3.C TRIGGERY ---
\subsection{Triggery}

W obecnej wersji bazy danych \textbf{nie zdefiniowano triggerów}. Zrezygnowano z triggerów na rzecz więzów integralności (FOREIGN KEY CASCADE) oraz procedur składowanych, co zapewnia lepszą wydajność i przejrzystość logiki bazy.

Integralność danych jest zapewniona przez:
\begin{itemize}[noitemsep]
    \item więzy klucza obcego (FK) z opcją ON DELETE CASCADE,
    \item ograniczenia CHECK walidujące wartości,
    \item ograniczenia DEFAULT dla wartości domyślnych,
    \item procedurę składowaną do bezpiecznego wstawiania/aktualizacji danych.
\end{itemize}

% --- 3.D PROCEDURY ---
\subsection{Procedury składowane}

\subsubsection{Procedura: UpsertCompany}

\textbf{Opis:} Procedura służąca do wstawiania nowej firmy lub aktualizacji istniejącej na podstawie unikalnego identyfikatora \texttt{mongo\_id}. Implementuje wzorzec UPSERT (UPDATE or INSERT).

\textbf{Parametry wejściowe:}
\begin{itemize}[noitemsep]
    \item \texttt{@mongo\_id} (NVARCHAR(50)) --- unikalny identyfikator z MongoDB (wymagany)
    \item \texttt{@name} (NVARCHAR(255)) --- nazwa firmy (wymagany)
    \item \texttt{@permalink} (NVARCHAR(255)) --- unikalny identyfikator URL (wymagany)
    \item \texttt{@category\_code} (NVARCHAR(100)) --- kategoria działalności (opcjonalny)
    \item \texttt{@description} (NVARCHAR(MAX)) --- opis firmy (opcjonalny)
    \item \texttt{@number\_of\_employees} (INT) --- liczba pracowników (opcjonalny)
    \item \texttt{@founded\_year} (INT) --- rok założenia (opcjonalny)
\end{itemize}

\textbf{Działanie:}
\begin{enumerate}[noitemsep]
    \item Sprawdza, czy firma o podanym \texttt{mongo\_id} już istnieje w bazie.
    \item Jeśli istnieje --- aktualizuje dane (UPDATE) i ustawia \texttt{updated\_at = GETDATE()}.
    \item Jeśli nie istnieje --- wstawia nowy rekord (INSERT).
    \item Wyświetla komunikat o wykonanej operacji.
\end{enumerate}

\begin{lstlisting}[caption={Procedura UpsertCompany}]
CREATE OR ALTER PROCEDURE crunchbase.UpsertCompany
    @mongo_id NVARCHAR(50),
    @name NVARCHAR(255),
    @permalink NVARCHAR(255),
    @category_code NVARCHAR(100) = NULL,
    @description NVARCHAR(MAX) = NULL,
    @number_of_employees INT = NULL,
    @founded_year INT = NULL
AS
BEGIN
    SET NOCOUNT ON;
    
    IF EXISTS (SELECT 1 FROM crunchbase.Company 
               WHERE mongo_id = @mongo_id)
    BEGIN
        UPDATE crunchbase.Company
        SET name = @name,
            category_code = @category_code,
            description = @description,
            number_of_employees = @number_of_employees,
            founded_year = @founded_year,
            updated_at = GETDATE()
        WHERE mongo_id = @mongo_id;
        
        PRINT 'Firma zaktualizowana: ' + @name;
    END
    ELSE
    BEGIN
        INSERT INTO crunchbase.Company 
            (mongo_id, name, permalink, category_code, 
             description, number_of_employees, founded_year)
        VALUES 
            (@mongo_id, @name, @permalink, @category_code, 
             @description, @number_of_employees, @founded_year);
        
        PRINT 'Firma dodana: ' + @name;
    END
END
\end{lstlisting}

\textbf{Zastosowanie:} Procedura umożliwia bezpieczne dodawanie i aktualizowanie firm bez ryzyka duplikacji danych. Jest wykorzystywana przez użytkownika \texttt{Emp} do zarządzania danymi.

% --- 3.E FUNKCJE ---
\subsection{Funkcje użytkownika}

\subsubsection{Funkcja skalarna: GetTotalFunding}

\textbf{Opis:} Funkcja skalarna zwracająca łączną sumę finansowania pozyskanego przez firmę we wszystkich rundach finansowania.

\textbf{Parametry wejściowe:}
\begin{itemize}[noitemsep]
    \item \texttt{@company\_id} (INT) --- identyfikator firmy
\end{itemize}

\textbf{Zwracana wartość:} \texttt{DECIMAL(18,2)} --- suma finansowania (0 jeśli brak danych)

\begin{lstlisting}[caption={Funkcja GetTotalFunding}]
CREATE OR ALTER FUNCTION crunchbase.GetTotalFunding(
    @company_id INT
)
RETURNS DECIMAL(18,2)
AS
BEGIN
    DECLARE @total DECIMAL(18,2);
    
    SELECT @total = ISNULL(SUM(raised_amount), 0)
    FROM crunchbase.FundingRound
    WHERE company_id = @company_id;
    
    RETURN @total;
END
\end{lstlisting}

\textbf{Zastosowanie:} Funkcja jest wykorzystywana w widoku \texttt{vw\_CompanyOverview} do prezentacji podsumowania finansowania każdej firmy. Może być również używana bezpośrednio w zapytaniach SELECT.

\textbf{Przykład użycia:}
\begin{lstlisting}
SELECT name, crunchbase.GetTotalFunding(company_id) AS funding
FROM crunchbase.Company
ORDER BY funding DESC;
\end{lstlisting}

% --- 3.F WIDOKI ---
\subsection{Widoki}

\subsubsection{Widok: vw\_CompanyOverview}

\textbf{Opis:} Widok prezentujący podsumowanie informacji o firmach wraz z obliczonym łącznym finansowaniem i liczbą produktów oraz rund finansowania.

\begin{lstlisting}[caption={Widok vw\_CompanyOverview}]
CREATE OR ALTER VIEW crunchbase.vw_CompanyOverview
AS
SELECT 
    c.company_id,
    c.name,
    c.category_code,
    c.founded_year,
    c.number_of_employees,
    c.homepage_url,
    crunchbase.GetTotalFunding(c.company_id) AS total_funding,
    (SELECT COUNT(*) FROM crunchbase.FundingRound fr 
     WHERE fr.company_id = c.company_id) AS funding_rounds_count,
    (SELECT COUNT(*) FROM crunchbase.Product p 
     WHERE p.company_id = c.company_id) AS products_count
FROM crunchbase.Company c;
\end{lstlisting}

\textbf{Kolumny widoku:}
\begin{itemize}[noitemsep]
    \item \texttt{company\_id} --- identyfikator firmy
    \item \texttt{name} --- nazwa firmy
    \item \texttt{category\_code} --- kategoria działalności
    \item \texttt{founded\_year} --- rok założenia
    \item \texttt{number\_of\_employees} --- liczba pracowników
    \item \texttt{homepage\_url} --- strona główna firmy
    \item \texttt{total\_funding} --- łączne finansowanie (obliczone przez funkcję)
    \item \texttt{funding\_rounds\_count} --- liczba rund finansowania
    \item \texttt{products\_count} --- liczba produktów
\end{itemize}

\textbf{Zastosowanie:} Widok umożliwia szybki przegląd najważniejszych informacji o firmach bez konieczności pisania złożonych zapytań. Jest udostępniony użytkownikowi \texttt{Guest} do przeglądania danych.

\textbf{Przykład użycia:}
\begin{lstlisting}
SELECT * FROM crunchbase.vw_CompanyOverview
ORDER BY total_funding DESC;
\end{lstlisting}

\newpage
% ============================================================================
% ROZDZIAŁ 4: ROLE, UPRAWNIENIA I UŻYTKOWNICY
% ============================================================================
\section{Role, uprawnienia i użytkownicy}

\subsection{Konfiguracja bazy dla Contained Users}

Baza danych została skonfigurowana z opcją \texttt{CONTAINMENT = PARTIAL}, co umożliwia tworzenie użytkowników przechowywanych lokalnie w bazie (contained users). Użytkownicy ci nie wymagają loginów na poziomie serwera.

\begin{lstlisting}[caption={Konfiguracja Contained Database Authentication}]
-- Wlaczenie zaawansowanych opcji
EXEC sp_configure 'show advanced options', 1;
RECONFIGURE;

-- Wlaczenie contained database authentication
EXEC sp_configure 'contained database authentication', 1;
RECONFIGURE;

-- Utworzenie bazy z CONTAINMENT = PARTIAL
CREATE DATABASE CompanyDB CONTAINMENT = PARTIAL;
\end{lstlisting}

\subsection{Role w bazie danych}

W bazie zdefiniowano trzy role odpowiadające różnym poziomom dostępu:

\begin{table}[H]
\centering
\caption{Role zdefiniowane w bazie danych}
\begin{tabular}{|l|l|p{7cm}|}
\hline
\textbf{Rola} & \textbf{Poziom} & \textbf{Uprawnienia} \\
\hline
AdminRole & Pełny & Członek roli \texttt{db\_owner} --- pełne uprawnienia do wszystkich obiektów bazy \\
\hline
EmpRole & Ograniczony & Uprawnienie EXECUTE na schemacie \texttt{crunchbase} --- może wykonywać procedury składowane \\
\hline
GuestRole & Minimalny & Uprawnienie SELECT na widoku \texttt{vw\_CompanyOverview} --- tylko odczyt przez widok \\
\hline
\end{tabular}
\end{table}

\begin{lstlisting}[caption={Tworzenie ról i nadawanie uprawnień}]
-- Rola administratora
CREATE ROLE AdminRole;
ALTER ROLE db_owner ADD MEMBER AdminRole;

-- Rola pracownika
CREATE ROLE EmpRole;
GRANT EXECUTE ON SCHEMA::crunchbase TO EmpRole;

-- Rola goscia
CREATE ROLE GuestRole;
GRANT SELECT ON crunchbase.vw_CompanyOverview TO GuestRole;
\end{lstlisting}

\subsection{Użytkownicy (Contained Users)}

W bazie utworzono trzech użytkowników lokalnych (contained users):

\begin{table}[H]
\centering
\caption{Użytkownicy w bazie danych}
\begin{tabular}{|l|l|l|p{5cm}|}
\hline
\textbf{Użytkownik} & \textbf{Hasło} & \textbf{Rola} & \textbf{Możliwości} \\
\hline
Admin & Admin & AdminRole & Pełny dostęp do bazy, tworzenie/modyfikacja/usuwanie obiektów \\
\hline
Emp & Emp & EmpRole & Wykonywanie procedur składowanych (np. UpsertCompany) \\
\hline
Guest & Guest & GuestRole & Przeglądanie danych tylko przez widok \\
\hline
\end{tabular}
\end{table}

\begin{lstlisting}[caption={Tworzenie użytkowników contained}]
-- Administrator
CREATE USER Admin WITH PASSWORD = 'Admin';
ALTER ROLE AdminRole ADD MEMBER Admin;

-- Pracownik
CREATE USER Emp WITH PASSWORD = 'Emp';
ALTER ROLE EmpRole ADD MEMBER Emp;

-- Gosc
CREATE USER Guest WITH PASSWORD = 'Guest';
ALTER ROLE GuestRole ADD MEMBER Guest;
\end{lstlisting}

\subsection{Sposób logowania jako Contained User}

Aby zalogować się jako contained user w SQL Server Management Studio:

\begin{enumerate}
    \item Server name: \texttt{localhost} (lub nazwa serwera)
    \item Authentication: \texttt{SQL Server Authentication}
    \item Login: \texttt{Admin} / \texttt{Emp} / \texttt{Guest}
    \item Password: \texttt{Admin} / \texttt{Emp} / \texttt{Guest}
    \item Options $\rightarrow$ Connect to database: \texttt{CompanyDB}
\end{enumerate}

\subsection{Testowanie uprawnień}

\textbf{Test użytkownika Admin:}
\begin{lstlisting}
-- Logowanie jako Admin - moze wszystko
SELECT * FROM crunchbase.Company;
INSERT INTO crunchbase.Company (mongo_id, name, permalink) 
VALUES ('test123', 'Test', 'test');
EXEC crunchbase.UpsertCompany 'test456', 'Test2', 'test2';
\end{lstlisting}

\textbf{Test użytkownika Emp:}
\begin{lstlisting}
-- Logowanie jako Emp - moze wykonac procedure
EXEC crunchbase.UpsertCompany 'test789', 'Test3', 'test3';

-- Nie moze bezposrednio modyfikowac tabel
SELECT * FROM crunchbase.Company; -- BLAD!
\end{lstlisting}

\textbf{Test użytkownika Guest:}
\begin{lstlisting}
-- Logowanie jako Guest - moze tylko przegladac widok
SELECT * FROM crunchbase.vw_CompanyOverview;

-- Nie moze nic wiecej
SELECT * FROM crunchbase.Company; -- BLAD!
EXEC crunchbase.UpsertCompany 'x', 'X', 'x'; -- BLAD!
\end{lstlisting}

\newpage
% ============================================================================
% ROZDZIAŁ 5: UWAGI KOŃCOWE
% ============================================================================
\section{Uwagi końcowe}

\subsection{Napotkane problemy}

Podczas realizacji projektu napotkano następujące wyzwania:

\begin{enumerate}
    \item \textbf{Struktura zagnieżdżona JSON} --- dokumenty JSON zawierały głęboko zagnieżdżone obiekty (np.\ \texttt{investments} wewnątrz \texttt{funding\_rounds}). Wymagało to użycia wielokrotnego \texttt{CROSS APPLY OPENJSON}.
    
    \item \textbf{Polimorficzne inwestycje} --- inwestorem może być osoba (\texttt{person}), organizacja finansowa (\texttt{financial\_org}) lub firma (\texttt{company}). Rozwiązano przez trzy opcjonalne klucze obce w tabeli \texttt{Investment}.
    
    \item \textbf{Referencje do nieistniejących firm} --- niektórzy konkurenci i przejęte firmy nie istnieją w danych źródłowych (6 firm). Zachowano ich nazwy i permalinki jako tekst, z opcjonalnym kluczem obcym jeśli firma istnieje w bazie.
    
    \item \textbf{Contained users} --- wymagana była dodatkowa konfiguracja serwera (\texttt{sp\_configure}) przed utworzeniem użytkowników lokalnych.
\end{enumerate}

\subsection{Elementy zrealizowane}

\begin{itemize}[noitemsep]
    \item[$\checkmark$] 18 tabel w schemacie \texttt{crunchbase} z pełną strukturą więzów integralności
    \item[$\checkmark$] Klucze główne (PK) i obce (FK) dla wszystkich tabel
    \item[$\checkmark$] Ograniczenia CHECK, UNIQUE i DEFAULT
    \item[$\checkmark$] 3 indeksy nieklastrowe na często używanych kolumnach
    \item[$\checkmark$] 1 procedura składowana (UpsertCompany)
    \item[$\checkmark$] 1 funkcja skalarna (GetTotalFunding)
    \item[$\checkmark$] 1 widok (vw\_CompanyOverview)
    \item[$\checkmark$] 3 role z różnymi uprawnieniami
    \item[$\checkmark$] 3 contained users (Admin, Emp, Guest)
    \item[$\checkmark$] Skrypt importu danych z JSON (OPENJSON)
    \item[$\checkmark$] Normalizacja do 3NF
\end{itemize}

\subsection{Pliki projektu}

\begin{table}[H]
\centering
\caption{Pliki wchodzące w skład projektu}
\begin{tabular}{|l|p{9cm}|}
\hline
\textbf{Plik} & \textbf{Opis} \\
\hline
01\_struktura.sql & Tworzenie bazy danych, schematu i wszystkich 18 tabel z ograniczeniami \\
\hline
02\_obiekty.sql & Procedura składowana, funkcja skalarna i widok \\
\hline
03\_uzytkownicy.sql & Role i contained users z uprawnieniami \\
\hline
04\_import.sql & Import danych z pliku JSON do wszystkich tabel \\
\hline
05\_testy.sql & Zapytania testowe i demonstracyjne (SELECT, JOIN, agregacje) \\
\hline
Raport\_CompanyDB.pdf & Niniejszy dokument \\
\hline
\end{tabular}
\end{table}

\subsection{Kolejność uruchamiania skryptów}

\begin{enumerate}
    \item \texttt{01\_struktura.sql} --- tworzenie bazy i tabel
    \item \texttt{02\_obiekty.sql} --- procedura, funkcja, widok
    \item \texttt{03\_uzytkownicy.sql} --- role i użytkownicy
    \item \texttt{04\_import.sql} --- import danych z JSON
    \item \texttt{05\_testy.sql} --- zapytania testowe (opcjonalnie, do demonstracji)
\end{enumerate}

\textbf{Uwaga:} W pliku \texttt{04\_import.sql} należy zmodyfikować ścieżkę do pliku JSON odpowiednio do lokalizacji na dysku.

\vspace{1cm}
\hrule
\vspace{0.5cm}
\begin{center}
\textit{Koniec dokumentu}
\end{center}

\end{document}
